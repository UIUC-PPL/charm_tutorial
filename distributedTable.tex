\begin{frame}[fragile]

  \frametitle{Power of Asynchrony}
  \framesubtitle{Example}

  \begin{itemize}
    \item Consider the following problem:
    \begin{itemize}
      \item A large number of key-value pairs are distributed on several (hundred) processors  (or chares)
      \pause
      \item Each chare needs to get some subset of these values before they can proceed to the next phase of the computation
      \pause 
      \item The set of keys needed are not known in advance: they are determined based on the input data
    \end{itemize}
  \end{itemize}
\end{frame}

\begin{frame}[fragile]
  \frametitle{Structured dagger version}
  \begin{lstlisting}[basicstyle=\footnotesize]
  for i=0,n {
      keys[i] = compute i’th key;
      KeyValues[keys[i]/B].requestValue( keys[i], thisProxy, i); 
  }
  \end{lstlisting}
  \pause
  \begin{lstlisting}[basicstyle=\footnotesize]
  for i= 0,n 
      when response(i,value)
          serial { values[i] = value;}

  // next phase of computation that’s uses the keys and values.
  \end{lstlisting}
  \pause
  \begin{lstlisting}[basicstyle=\footnotesize]
  KeyValueClass::requestValue(int key, Cproxy_Client c, int ref )
  {  
      ValueType v = localTable[key];
      c.response(ref, v); 
  }
  \end{lstlisting}
\end{frame}

\begin{frame}[fragile]

  \frametitle{Threaded EP version}

  \begin{lstlisting}[basicstyle=\footnotesize]
  for i=0,n {
      keys[i] = compute i’th key;
      futures[i] = CkCreateFuture();
      KeyValues[keys[i]/B].requestViaFuture(futures[i], keys[i]); 
  }
  \end{lstlisting}
  \pause
  \begin{lstlisting}[basicstyle=\footnotesize]
  for i= 0,n {
      ValueMsg * m = CkWaitFuture(futures[i]);
      values[i] = m->value;
  }
  // next phase of computation that’s uses the keys and values.
  \end{lstlisting}
  \pause
  \begin{lstlisting}[basicstyle=\footnotesize]
  KeyValueClass::requestViaFuture(CkFuture f , int key)
  {  
      ValueType v = localTable[key];
      ValueMsg * m = new ValueMsg;
      m->value = v;
      CkSendToFuture(f,m);
  }
  \end{lstlisting}
\end{frame}

\begin{frame}[fragile]

  \frametitle{MPI version}
  \framesubtitle{Can be hard and inefficient}

  \begin{lstlisting}[basicstyle=\footnotesize]
  for i=0,n {
      keys[i] = compute i’th key; 
      MPI send request to processor keys[i]/B
  }
  for i= 0,n {
      recv msg  containing the reference and value and store it
  }
  // next phase of computation that’s uses the keys and values.
  \end{lstlisting}
  \pause
  \textcolor{red}{But this does not work because you have to process 
  requests as well..And you don’t know how many requests there are.}
  \pause
  \\
  \vspace{.2cm}
  Other alternatives are possible, e.g. using 2 AllToAlls; but can be inefficient.
\end{frame}


