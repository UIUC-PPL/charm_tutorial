\documentclass{beamer}

\usetheme[secheader]{Boadilla}
\usecolortheme{seahorse}
\usepackage[latin1]{inputenc}

\title{Charm++ Tutorial}
\author{L.V. Kale, Eric Bohm}
\date{November 11, 20012}
\institute[2012]{Parallel Programming Lab, UIUC}

\begin{document}

\frame{\titlepage}

\section[Outline]{}
\frame{\tableofcontents}
\section{Introduction}
\section{Model motivation and benefits: broad brush}
\section{Chares, recursive, grainsize}
\section{Basic Chare Array}
\section{SdAg}
\section{Pmodels jacobi}
\section{Object based decompostion: natural objects, use multiple chare arrays, stencil (overlap, cache benefit, ), leanmd (md design), Openatom slide}
\section{Modularity and compositionality: (phil: sanjay’s slides)}
\subsection{examples rocket, namd+pme}
\section{Prioritization (Ram)}
\section{Load balancing and mapping (eric?)}
\section{Fault Tolerance (Esteban)}
\section{Heterogeneity (??)}
\section{Groups and NodeGroups: (Lukasz) }
\section{ChaNGa (graphically) using groups for cache}
\section{Mesh Streamer (??)}
\section{Conditional copying, and within-node sharing (pritish?)}
\end{document}
