\documentclass{article}

\usepackage{amsmath}

\title{MP0}
\author{CS 598: Parallel Migratable Objects}

\begin{document}
\maketitle

For this MP, you will create a simple Charm++ program that calculates the
determinants of 2x2 and 3x3 matrices in parallel (individual determinant
calculations should not be sequentialized).

You should define two chare classes, a mainchare named \texttt{Main} and a
chare that calculates determinants named \texttt{DeterminantChare}. The
mainchare should send $n$ randomly generated 2x2 matrices and $m$ randomly
generated 3x3 matrices to different instances of \texttt{DeterminantChare}
($n+m$ instances total), where $n$ and $m$ are command line parameters read by
the mainchare in that order. The matrices sent to the \texttt{DeterminantChare}
should be generated using \texttt{rand()}, seeding the random number generator
initially with \texttt{srand(29)}, and should be passed as a one-dimensional
C++ array of integers. Each \texttt{DeterminantChare} will calculate the
determinant of the matrix it receives, using the following formula for 2x2
matrices:

\[
\begin{vmatrix}
  a & b \\
  c & d \\
\end{vmatrix} = ad - cb
\]%
and the following for 3x3 matrices:
\[
\begin{vmatrix}
  a & b & c \\
  d & e & f \\
  g & h & i \\
\end{vmatrix} = aei+bfg+cdh-ceg-bdi-afh
\]

Each \texttt{DeterminantChare} should then send the determinant back to the
mainchare for the matrix it received. The mainchare \texttt{Main} should then
print to \texttt{stdout} in the order it created the requests, the resulting
determinants. After printing all $n + m$ determinants to the screen, the
mainchare \texttt{Main} should call \texttt{CkExit()}, terminating the program.

You should run the program on the EWS cluster with $n = 5, m = 5$ and turn in
the output of this execution and the code.

\end{document}