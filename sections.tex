\begin{frame}[fragile]
\frametitle{Sections}
\framesubtitle{Motivation}
\begin{itemize}
 \item It is often convenient to define subcollections of elements within a
   chare array
   \begin{itemize}
   \item Example: rows or columns of a 2D chare array
   \item One may wish to perform collective operations on the subcollection
     (e.g. broadcast, reduction)
   \end{itemize}
 \item Sections are the standard subcollection construct in Charm++
\end{itemize}
\end{frame}

\begin{frame}[fragile]
\frametitle{Section Creation}
\begin{itemize}
 \item Through explicit enumeration:
\end{itemize}

\begin{lstlisting}
int k = 7;

std::vector<CkArrayIndex3D> elems; // add array indices
for (int i = 5; i < 10; i++)
  for (int j = 10; j < 20; j++)
      elems.push_back(CkArrayIndex3D(i, j, k));

CProxySection_Hello proxy =
  CProxySection_Hello::ckNew(helloArrayID, &elems[0], elems.size());
\end{lstlisting}
\end{frame}

\begin{frame}[fragile]
\frametitle{Section Creation}
\begin{itemize}
 \item Through index range specification:
\end{itemize}

\begin{lstlisting}
CProxySection_Hello proxy =
  CProxySection_Hello::ckNew(helloArrayID, 0, 9, 1, 0, 19, 2, 0, 29, 2);
\end{lstlisting}
\end{frame}

\begin{frame}[fragile]
\frametitle{Section Class Generation}
\begin{itemize}
 \item Section proxy classes are automatically generated for each chare and group defined in the ci file
 \item Placed into decl.h and def.h files
   \begin{itemize}
   \item Take a look at the CProxySection classes and their member functions in
     the decl.h and def.h for your MPs
   \end{itemize}
\end{itemize}
\end{frame}

\begin{frame}[fragile]
\frametitle{Using Sections}
\begin{lstlisting}
CProxySection_Hello proxy;

// section broadcast
proxy.someEntry(...) 

// send to the first element in the section
proxy[0].someEntry(...) 
\end{lstlisting}
\end{frame}

% \begin{frame}[fragile]
% \frametitle{CkMulticast}
% \begin{itemize}
% \item Charm++ library for high performance broadcasts and reductions over
%   sections
% \end{itemize}
% \begin{lstlisting}
% # compile and install the CkMulticast library
% cd charm/tmp
% make multicast
% # link CkMulticast library using -module
% charmc -o hello hello.o -module CkMulticast -language charm++
% \end{lstlisting}
% \end{frame}

% \begin{frame}[fragile]
% \frametitle{Delegation}
% \begin{itemize}
% \item A way to customize Charm++ send and receive behavior
% \item \code{CkDelegateMgr}
%   \begin{itemize}
%   \item "interface" class with virtual functions for sending messages to
%     chares, groups, nodegroups, arrays, sections
%   \end{itemize}
% \item Delegation Manager
%   \begin{itemize}
%   \item Any group which inherits from \code{CkDelegateMgr}
%   \item Specifies communication behavior by overriding one or more of the
%     default communication functions
%   \end{itemize}
% \end{itemize}
% \end{frame}

% \begin{frame}[fragile]
% \frametitle{Delegation}
% \begin{itemize}
% \item Achieved by associating a delegation manager to a proxy
%   \begin{itemize}
%   \item Different proxies for the same array or section may be associated with
%     different delegation managers with distinct implementations of
%     communication functions
%   \end{itemize}
% \end{itemize}
% \end{frame}

% \begin{frame}[fragile]
% \frametitle{\code{CkMulticastMgr}}
% \begin{itemize}
% \item A delegation manager which implements section broadcasts and reductions
%   \begin{itemize}
%   \item Overrides \code{CkDelegateMgr::ArraySectionSend}
%   \end{itemize}
% \end{itemize}

% \begin{lstlisting}
% CProxySection_Hello sectProxy = CProxySection_Hello::ckNew(...);
% CkGroupID mCastGrpId = CProxy_CkMulticastMgr::ckNew();
% CkMulticastMgr *mCastGrp = CProxy_CkMulticastMgr(mCastGrpId).ckLocalBranch();
% // initialize section proxy
% sectProxy.ckSectionDelegate(mCastGrp);
% // multicast via delegation library as before
% sectProxy.someEntry(...); 
% \end{lstlisting}
% \end{frame}

% \begin{frame}[fragile]
% \frametitle{Spanning Trees}
% \begin{itemize}
% \item CkMulticast implements tree algorithms for multicasts and reductions
%   \begin{itemize}
%   \item Messages are routed over a spanning tree of the section elements
%   \end{itemize}
% \item Default branching factor is 2
% \item Modifying branching factor:
% \end{itemize}

% \begin{lstlisting}
% CkGroupID mCastGrpId = 
%   CProxy_CkMulticastMgr::ckNew(3); // factor is 3
% \end{lstlisting}
% \end{frame}

% \begin{frame}[fragile]
% \frametitle{CkMulticast Messages}
% \begin{itemize}
% \item To use CkMulticast library, all multicast messages must inherit from
%   \code{CkMcastBaseMsg}
%   \begin{itemize}
%   \item \code{CkMcastBaseMsg} must be inherited from first
%   \item No parameter marshalling is allowed in entry methods used as targets of
%     multicast
%   \end{itemize}
% \end{itemize}

% \begin{lstlisting}
% struct HiMsg : public CkMcastBaseMsg, public CMessage_HiMsg {
%   int *data;
% };
% \end{lstlisting}
% \end{frame}

% \begin{frame}[fragile]
% \frametitle{Reductions: \code{CkSectionInfo}}
% \begin{itemize}
% \item An array element can be a member of multiple array sections
%   \begin{itemize}
%   \item It is necessary to disambiguate which array section reduction it is
%     participating in each time it contributes to one
%   \end{itemize}
% \item A data structure called \code{CkSectionInfo} is created by CkMulticastMgr
%   for each array section that the array element belongs to
%   \begin{itemize}
%   \item During a section reduction, the array element must pass the
%     \code{CkSectionInfo} as a parameter in the \code{contribute()}
%   \end{itemize}
% \end{itemize}
% \end{frame}

% \begin{frame}[fragile]
% \frametitle{Reductions with CkMulticast}
% \begin{lstlisting}
% CkSectionInfo cookie;
% void SayHi(HiMsg *msg) {
%   // update section cookie every time
%   CkGetSectionInfo(cookie, msg);
%   int data = thisIndex;
%   mcastGrp->contribute(sizeof(int), &data,
%                        CkReduction::sum_int, cookie);
% }
% \end{lstlisting}
% \end{frame}

% \begin{frame}[fragile]
% \frametitle{Callbacks}
% \begin{itemize}
% \item As with array reductions, a callback needs to be specified with each
%   contribute
%   \begin{itemize}
%   \item OR a default callback should be specified using setReductionClient
%   \end{itemize}
% \end{itemize}
% \end{frame}

\begin{frame}[fragile]
\frametitle{Example: Matrix Multiplication}
\begin{itemize}
\item Inputs: 2D chare arrays A, B of matrix blocks
\item Output: 2D chare array C of matrix blocks
\item Elements of A and B multicast their blocks to a section comprising a row
  or column of C
\item Exercise: implement algorithm
\end{itemize}
\end{frame}