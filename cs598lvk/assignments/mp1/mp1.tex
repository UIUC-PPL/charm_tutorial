\documentclass{article}

\usepackage{amsmath}
\usepackage[colorlinks=true]{hyperref}

\title{MP1 \\ CS 598: Parallel Migratable Objects}
\author{Fall 2013}
\date{Due date: September 27, 10pm}

\begin{document}
\maketitle

\textbf{Description:} 

For this assignment you need to write a Charm++ program which does load
balancing of elements in a chare array using the given parallel prefix code in
the class slides or with your own parallel prefix code. \\
First, you need to create a chare array where each chare has a different number 
of values: randomly create a number between \texttt{min} and \texttt{max} where
min, max and chare array size are command line parameters. 
Use parallel prefix to
decide which chare array element should send how many values to which chare
array element and send those messages. In the end every chare in the chare array
need to have equal number of elements. You can either use QD to terminate or
your own method to detect if all the messages that needs to be received are
received. 

\textbf{Running your code:}
The program should have the following signature:
\textit{./charmrun /mp1 +p4 chare-array-size min max} \\
You should test your code on Taub cluster with upto x?? number of processors.

\textbf{Submission:}
Create an mp1 directory in your SVN repository folder and add your code into
that folder and check in the your code.
\begin{itemize}
\item  For each file F you create, that you want to check in, do: \\
        \textit{svn add F}\\
        and frequently (after you have modified F, and have the next better
        version) do:\\ 
        \textit{svn ci F}
\item  There will be a penalty for late submissions.
\end{itemize}

\end{document}
