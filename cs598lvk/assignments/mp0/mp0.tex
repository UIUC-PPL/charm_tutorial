\documentclass{article}

\usepackage{amsmath}
\usepackage[colorlinks=true]{hyperref}

\title{MP0}
\title{CS 598: Parallel Migratable Objects}
\author{Fall 2014}
\date{Due Date: September 4th, 10 PM CDT}

\begin{document}
\maketitle

This MP is an introductory assignment to learn how to build Charm++ and run
Charm++ programs. You will also learn how to use the \texttt{Projections}
performance analysis and visualization tool. \\

For learning Charm++ programming and concepts, please refer to the
\href{http://charm.cs.illinois.edu/manuals/html/charm++/}{Charm++ Manual},
the \href{http://charm.cs.illinois.edu/tutorial/}{Charm++ Tutorial},
and the \href{http://charmplusplus.org/}{Charm++ Webpage}.\\

\textbf{Step 1:} To run Charm++ programs, you need to have a Charm++ build
installed. If you want to build your own installation, follow the build
instructions \href{http://charm.cs.illinois.edu/manuals/html/charm++/A.html}{here}. Both
on EWS and Taub Caumpus cluster, there are two installations of Charm++ that can
be used instead of building Charm++ yourself. Path is the same in both machines. 
\begin{itemize}
\item /home/acun2/charm/net-linux-x86\_64
\item /home/acun2/charm/net-linux-x86\_64-smp\\
\end{itemize}
For this MP0, you only need to use EWS. Taub has a batch
system to submit jobs which will be required to use later, however for this MP
it is optional.\\
For EWS login instructions go to
\href{http://it.engineering.illinois.edu/ews/lab-information/ews-faq}{EWS FAQ
page.} \\
For Taub login instructions go to
\href{https://campuscluster.illinois.edu/user\_info/doc/}{Taub User Guide.} \\
\\
\textbf{Step 2:} An svn repository has been created for the class
\href{https://subversion.ews.illinois.edu/svn/fa13-cs598lvk/}{here} at EWS, where you can find the Charm++ code for MP0. \\
\begin{itemize}
\item Checkout the code from the repository by: \\
        \textit{svn co
        https://subversion.ews.illinois.edu/svn/fa13-cs598lvk/netID \\
        where netID is your actual netID}
\item A directory named mp0 should be created, go into that directory: \\
        \textit{cd netID/mp0}
\item Analyze the code and run the program with \textit{make test} command.
It should print "Total number of primes within the range [0 - 100000] is 9591." In the program Master chare fires k chares, i'th Worker chare fired is responsible for
computing number of primes between [ i*M ..(i+1)*M ]. They return the counts to
the Master chare. M and k are command line arguments. You can change these numbers in the \texttt{Makefile} and experiment.

\end{itemize}
    To run a program in Taub cluster you need to submit batch scripts, sample
    running script is added to your SVN repositories.  \\

\textbf{Step 3:}  After running the program, you will see \texttt{Projections}
logs created in the same directory. Anaylze them using the \texttt{Projections} tool, take a snapshot of the timeline of the program and turn it in. \texttt{Projections} can be
downloaded from \href{http://charm.cs.uiuc.edu/software}{here} and the
\texttt{Projections} manual can
be found
\href{http://charm.cs.illinois.edu/manuals/html/projections/manual-1p.html}
{here}. \\

\textbf{Submission:} \\
Submission will be done to SVN repositories.
\begin{itemize}
\item  For each file F you create, that you want to check in, do: \\
        \textit{svn add F}\\
        and frequently (after you have modified F, and have the next better
        version) do:\\ 
        \textit{svn ci F}
\item  There will be a penalty for late submissions.
\end{itemize}

\end{document}
