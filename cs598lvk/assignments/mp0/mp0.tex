\documentclass{article}

\usepackage{amsmath}
\usepackage[colorlinks=true]{hyperref}

\title{MP0}
\title{CS 598: Parallel Migratable Objects}
\author{Fall 2014}
\date{Due Date: September 4th, 10 PM CDT}

\begin{document}
\maketitle

This MP is an introductory assignment to learn how to build Charm++ and run
Charm++ programs. You will also learn how to use the \texttt{Projections}
performance analysis and visualization tool. \\

For learning Charm++ programming and concepts, please refer to the
\href{http://charm.cs.illinois.edu/manuals/html/charm++/}{Charm++ Manual},
the \href{http://charm.cs.illinois.edu/tutorial/}{Charm++ Tutorial},
and the \href{http://charmplusplus.org/}{Charm++ Webpage}.\\

\textbf{Step 1:} To run Charm++ programs, you need to have a Charm++ build installed.
On both EWS and Taub (the campus cluster), there are two installations of Charm++ that can be used.
The path is the same on both machines.
\begin{itemize}
\item /home/acun2/charm/net-linux-x86\_64
\item /home/acun2/charm/net-linux-x86\_64-smp\\
\end{itemize}

If you want to build your own installation, follow the build
instructions \href{http://charm.cs.illinois.edu/manuals/html/charm++/A.html}{here}.
For Linux we recommend the following build line:
\begin{itemize}
\item \texttt{./build charm++ net-linux-x86\_64 -j8}
\end{itemize}
For Mac we recommend the following build line:
\begin{itemize}
\item \texttt{./build charm++ net-darwin-x86\_64 -j8}
\end{itemize}
You can also run the interactive smartbuild script and pick your build options using \texttt{./build}.\\

For this MP0, you only need to use EWS. Taub has a batch
system to submit jobs which you will be required to use later, however for this MP
it is optional.\\
For EWS login instructions go to
\href{http://it.engineering.illinois.edu/ews/lab-information/ews-faq}{EWS FAQ page}.\\
For Taub login instructions go to
\href{https://campuscluster.illinois.edu/user\_info/doc/}{Taub User Guide}.\\
\\
\textbf{Step 2:} An svn repository for the class has been created
\href{https://subversion.ews.illinois.edu/svn/fa14-cs598lvk/}{here},
where you can find the Charm++ code for MP0.\\
\begin{itemize}
\item Checkout the code from the repository by:\\
        \texttt{svn co
        https://subversion.ews.illinois.edu/svn/fa14-cs598lvk/netID}\\
        where netID is your actual netID
\item A directory named mp0 should be created, go into that directory:\\
        \texttt{cd netID/mp0}
\item Analyze the code and run the program with the command \texttt{make test}.
It should print ``Total number of primes within the range [0 - 100000] is 9591.''
In the program, the Master chare creates k chares.
The i'th Worker chare created is responsible for computing the number of primes between [i*M .. (i+1)*M].
Each Worker chare returns their count to the Master chare.
M and k are command line arguments.
You can change these numbers in the \texttt{Makefile} and experiment.

\end{itemize}
    To run a program on Taub you need to submit batch scripts.
    A sample batch script is included in your SVN repositories.\\

\textbf{Step 3:}  After running the program, you will see \texttt{Projections}
logs created in the same directory.
Analyze them using the \texttt{Projections} tool,
take a snapshot of the timeline of the program, and turn it in.
\texttt{Projections} can be downloaded from \href{http://charm.cs.illinois.edu/software}{here} and the
\texttt{Projections} manual can be found
\href{http://charm.cs.illinois.edu/manuals/html/projections/manual-1p.html}
{here}.\\

\textbf{Submission:}
Submission will be done via your SVN repository.
\begin{itemize}
\item  For each file F you create and want to check in, do:\\
        \texttt{svn add F}
\item  And frequently (after you have modified F and have a newer version) do:\\
        \texttt{svn ci F}
\item  There will be a penalty for late submissions.
\end{itemize}

\end{document}
