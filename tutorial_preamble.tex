\usepackage{fancyvrb}
\usepackage{color, colortbl}
\usepackage{listings}
\usepackage{url}
\usepackage{array}
\usepackage{calc}
\usepackage{ctable}
\usepackage{amsmath}
\usepackage{cite}
\usepackage{graphicx}
\usepackage{listings}
\usepackage{xspace}
\usepackage{hyperref}
\usepackage{subfigure}
\usepackage{multicol}
\definecolor{lightgray}{rgb}{0.9,0.9,0.9}

\lstset{ %
language=C++,                % choose the language of the code
basicstyle=\tiny,       % the size of the fonts that are used for the code
%numbers=left,                   % where to put the line-numbers
numbers=none,                   % where to put the line-numbers
numberstyle=\scriptsize,      % the size of the fonts that are used for the line-numbers
stepnumber=1,                   % the step between two line-numbers. If it's 1 each line will be numbered
numbersep=15pt,                  % how far the line-numbers are from the code
backgroundcolor=\color{lightgray},  % choose the background color. You must add \usepackage{color}
%backgroundcolor=none,  % choose the background color. You must add \usepackage{color}
showspaces=false,               % show spaces adding particular underscores
showstringspaces=false,         % underline spaces within strings
showtabs=false,                 % show tabs within strings adding particular underscores
frame=single,	                % adds a frame around the code
%frame=none,	                % adds a frame around the code
tabsize=2,	                % sets default tabsize to 2 spaces
%captionpos=b,                   % sets the caption-position to bottom
captionpos=n,
%basicstyle=\small,
%basicstyle=\small\sffamily,
basicstyle=\sffamily\small,
%basicstyle=\ttfamily\small,
breaklines=true,                % sets automatic line breaking
breakatwhitespace=false,        % sets if automatic breaks should only happen at whitespace
columns=fullflexible,
title=\lstname,                 % show the filename of files included with \lstinputlisting; also try caption instead of title
escapeinside={\%*}{*)},          % if you want to add a comment within your code
morekeywords={chare,mainchare,module,mainmodule,entry,readonly,array,serial,for,when,if,then,else,overlap,while,forall,threaded,sync,message},
aboveskip=2pt,
belowskip=2pt,
lineskip=0pt,
xleftmargin=1em,
xrightmargin=1em,
%xleftmargin=10pt
abovecaptionskip=0pt,
belowcaptionskip=0pt,
}

\newcommand{\code}[1]{\colorbox{lightgray}{\texttt{#1}}}
\newcommand{\transition}[1]{\begin{frame}[plain]\begin{center}\LARGE #1\end{center}\end{frame}}
\DefineVerbatimEnvironment{codeverb}{Verbatim}{fontsize=\small}

\hypersetup{
    colorlinks,%
    citecolor=black,%
    filecolor=black,%
    linkcolor=black,%
    urlcolor=magenta
}

\usefonttheme{professionalfonts}
\usetheme{Boadilla}
%\usetheme{Warsaw}
\usecolortheme{beaver}
%\AtBeginSubsection[]
%{
%    \begin{frame}{Outline}
%        \tableofcontents[currentsection,currentsubsection]
%    \end{frame}
%}

\AtBeginSection[]{
%  \setbeamercolor{section in toc shaded}{use=structure,fg=structure.fg}
%  \setbeamercolor{section in toc}{fg=mycolor}
%  \setbeamercolor{subsection in toc shaded}{fg=black}
%  \setbeamercolor{subsection in toc}{fg=mycolor}
  \frame<beamer>{\begin{multicols}{2}
  \frametitle{Outline}
  \setcounter{tocdepth}{2}  
%  \tableofcontents[currentsection,subsections]
  \tableofcontents[currentsection,currentsubsection]
\end{multicols} 
 }
}

