\begin{frame}[fragile]
  \frametitle{Declaring a Chare Array}
  \texttt{.ci} file:
  \begin{lstlisting}
    array [1d] foo {
      entry foo(); // constructor
      // ... entry methods ...
    }
    array [2d] bar {
      entry bar();  // constructor
      // ... entry methods ...
    }
  \end{lstlisting}
  \texttt{.C} file:

  \begin{lstlisting}
    struct foo : public CBase_foo {
      foo() { }
      foo(CkMigrateMessage*) { }
    };
    struct bar : public CBase_bar {
      bar() { }
      bar(CkMigrateMessage*) { }
    };
  \end{lstlisting}
\end{frame}

\begin{frame}[fragile]
  \frametitle{Constructing a Chare Array}
  \begin{itemize}
    \item Constructed much like a regular chare
    \item The size of each dimension is passed to the constructor
  \end{itemize}
  \begin{lstlisting}
    void someMethod() {
       CProxy_foo::ckNew(10);
       CProxy_bar::ckNew(5, 5);
    }
  \end{lstlisting}
  \begin{itemize}
  \item The proxy may be retained:
  \end{itemize}
  \begin{lstlisting}
    CProxy_foo myFoo = CProxy_foo::ckNew(10);
  \end{lstlisting}
  \begin{itemize}
  \item The proxy represents the entire array, and may be indexed to obtain a
    proxy to an individual element in the array
  \end{itemize}
  \begin{lstlisting}
    CProxyElement_foo elm = myFoo[5];
    elm.invokeEntry();
    myFoo[4].invokeEntry();
  \end{lstlisting}
\end{frame}

\begin{frame}[fragile]
  \frametitle{\code{thisIndex}}
  \begin{itemize}
  \item 1d: \code{thisIndex} returns the index of the current chare array element
  \item 2d: \code{thisIndex.x} and \code{thisIndex.y} returns the indices of
    the current chare array element
  \end{itemize}
  \texttt{.ci} file:
  \begin{lstlisting}
    array [1d] foo {
      entry foo();
    }
  \end{lstlisting}

  \texttt{.C} file:
  \begin{lstlisting}
    struct foo : public CBase_foo {
      foo() {
        CkPrintf("array index = %d", thisIndex);
      }
    };
  \end{lstlisting}

\end{frame}

\removeForTutorial{
\begin{frame}[fragile]
  \frametitle{Chare Array: Hello Example }
  \lstinputlisting{code/arrayHello.ci}
\end{frame}

\begin{frame}[fragile]
  \frametitle{Chare Array: Hello Example }
  \lstinputlisting[basicstyle=\scriptsize]{code/arrayHello.cpp}
\end{frame}

\begin{frame}[fragile]
   \frametitle{Hello World Array Projections Timeline View}\scriptsize
  \begin{itemize}
    \item Add \texttt{-tracemode projections} to link line to enable tracing
    \item Run Projections tool to load trace log files and visualize performance
  \end{itemize}
  \begin{center} \includegraphics[width=0.99\textwidth]{figures/arrayHelloTimeline} \end{center}
  \begin{itemize}
   \item arrayHello on BG/Q 16 Nodes, mode c16, 1024 elements (4 per process)
  \end{itemize}
\end{frame}



}

\begin{frame}[fragile]
  \frametitle{Collections of Objects: Runtime Service}
  \begin{itemize}
    \item System knows how to `find' objects efficiently: $(collection, index) \to processor$
    \item Applications can specify a mapping, or use simple
      runtime-provided options (e.g. blocked, round-robin)
    \item Distribution can be static, or dynamic!
    \item Key abstraction: application logic doesn't change, even
      though performance might
  \end{itemize}
\end{frame}

\begin{frame}[fragile]
  \frametitle{Collections of Objects: Runtime Service}
  \begin{itemize}
    \item Can develop and test logic in objects separately from their distribution
    \item Separation in time: make it work, then make it fast
    \item Division of labor: domain specialist writes object code, computationalist writes mapping
    \item Portability: different mappings for different systems, scales, or configurations
    \item Shared progress: improved mapping techniques can benefit existing code
  \end{itemize}
\end{frame}

\begin{frame}[fragile]
  \frametitle{Collections of Objects}
  \begin{center}\includegraphics[width=0.9\textwidth]{figures/elements2.pdf}\end{center}
\end{frame}

\begin{frame}[fragile]
  \frametitle{Collective Communication Operations}
  \begin{itemize}
    \item Point-to-point operations involve only two objects
    \item Collective operations that involve a collection of objects
    \item Broadcast: calls a method in each object of the array
    \item Reduction: collects a contribution from each object of the array
    \item A spanning tree is used to send/receive data
  \end{itemize}
    \begin{center} \includegraphics[width=0.5\textwidth]{figures/spanningTree.pdf} \end{center}
\end{frame}


\begin{frame}[fragile]
  \frametitle{Broadcast}
  \begin{itemize}
    \item A message to each object in a collection
    \item The chare array proxy object is used to perform a broadcast
    \item It looks like a function call to the proxy object
    \item From the main chare:
    \begin{lstlisting}
CProxy_Hello helloArray = CProxy_Hello::ckNew(helloArraySize);
helloArray.foo();
    \end{lstlisting}
    \item From a chare array element that is a member of the same array:
     \begin{lstlisting}
 thisProxy.foo()
    \end{lstlisting}
  \end{itemize}
\end{frame}

\begin{frame}[fragile]
  \frametitle{Reduction}
  \begin{itemize}
  \item Combines a set of values: sum, max, aggregate
  \item Usually reduces the set of values to a single value
  \item Combination of values requires an operator
  \item The operator must be commutative and associative
  \item Each object calls \code{contribute} in a reduction
  \end{itemize}
\end{frame}

\begin{frame}[fragile]
  \frametitle{Reduction: Example}
  \lstinputlisting{code/reductionMainTarget.ci}
\end{frame}

\begin{frame}[fragile]
  \frametitle{Reduction: Example}
  \lstinputlisting[basicstyle=\tiny]{code/reductionMainTarget.cpp}
Output:
  \begin{lstlisting}[basicstyle=\tiny]
value: 1176
Program finished.
  \end{lstlisting}
\end{frame}

\removeForClass{
\begin{frame}[fragile]
  \frametitle{Quick Hands-on}
  \begin{itemize}
  \item Log onto your vesta account.
  \item Obtain the following code:\\ git clone git://charm.cs.uiuc.edu/users/tutorial\_exercise
  \item Read the README.
  \item Change to toy directory, and read assignment.txt.
  \item Uncomment the CHARMC declaration at top of Makefile and make.
  \item ./charmrun -A $<$your\_account$>$ +p4 ./hello 16.
  \item Modify paramter to be an array instead of int.
  \end{itemize}
\end{frame}
}

%comment for removeForClass macro to work
