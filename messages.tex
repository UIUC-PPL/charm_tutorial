%%%%%%%%%%%%%%%%%%%%%%%%%%%%%%%%%%%%
%%                                %%
%%       Messages, Priorities     %%
%%         Groups, Nodegroups     %%
%%%%%%%%%%%%%%%%%%%%%%%%%%%%%%%%%%%%

\begin{frame}[fragile]
  \frametitle{Messages}
  \begin{itemize}
    \item A message is a struct | class that inherits from a system-defined class
    \item It provides explicit control for the application over allocation, reuse, and scope
    \begin{itemize}
      \item Marshalled parameters go out of scope at the end of the entry method
      \item Messages are explicitely deleted by the application
    \end{itemize}
    \item In the ``.ci'' file:
      \begin{itemize}
         \begin{lstlisting}
          message MyMsgType;
         \end{lstlisting}
         \item The interface translator produces code for a class named
         \begin{lstlisting}
          CMessage_MyMsgType
         \end{lstlisting}
      \end{itemize} 
      \item In the “.h” or “.C” file, define
      \begin{lstlisting}
       MsgTypeName  class
      \end{lstlisting}
  \end{itemize}
\end{frame}

\begin{frame}[fragile]
  \frametitle{Variable-Size Messages}
  \framesubtitle{Declaration}
  \begin{itemize}
    \item In the “.ci” file, declare the name and type of variable length arrays
    \begin{lstlisting}
      message MyVarsizeMsg {
        int arr1[];
        MyStruct arr2[];
      };
    \end{lstlisting}
    \item Corresponding definition in .h file
    \begin{lstlisting}
     class MyVarsizeMsg : public CMessage_MyVarsizeMsg {
       // variable-length arrays
       int *arr1;
       MyStruct *arr2;
     };
    \end{lstlisting}
  \end{itemize}
\end{frame}

\begin{frame}[fragile]
  \frametitle{Variable-Size Messages}
  \framesubtitle{Allocation}
  \begin{itemize}
    \item Use optional arguments in brackets for variable size arrays
    \item Sizes used in-order to allocate arrays
  \end{itemize}
  \begin{lstlisting}
     MessageType *msgptr = new (int sz1, int sz2, ... ) 
                                MessageType(constructor arguments);

     MyVarsizeMsg *msg = new (10,  7)        
                                      MyVarsizeMsg(<constructor args>);
  \end{lstlisting}
  \begin{itemize}
    \item This defines arr1 to be an integer array of length 10, and arr2 to be MyStruct array of length 7
    \item Charm forces them to be in a contiguous memory segment
  \end{itemize}
\end{frame}

\begin{frame}[fragile]
  \frametitle{Variable-Size Messages}
  \framesubtitle{Allocation by Array}
  \begin{itemize}
    \item Alternatively, one can use an array of integers to specify size of variable arrays
    \begin{lstlisting}
      int sizes[2]; 
      sizes[0] = 10; sizes[1] = 7
      MyVarsizeMsg *msg = new (sizes)        
                                      MyVarsizeMsg(<constructor args>);
    \end{lstlisting}
    \item Messages passed to Charm belong to Charm – deleted or reused by Charm after sending
    \item Message delivered by Charm belongs to user – can be reused or deleted
  \end{itemize}
\end{frame}

\begin{frame}[fragile]
  \frametitle{Changing Message Order}
  \begin{itemize}
    \item To set the queueing strategy for execution of received events
    \begin{itemize}
      \item \code{void CkSetQueueing(MsgType message, int queueingtype)}
    \end{itemize}
    \item Following queueing types can be set
    \begin{itemize}
      \item \code{CK\_QUEUEING\_FIFO} : FIFO ordering (default)
      \item \code{CK\_QUEUEING\_LIFO} : LIFO ordering
      \item \code{CK\_QUEUEING\_IFIFO} : FIFO ordering with integer priority
      \item \code{CK\_QUEUEING\_ILIFO} : LIFO ordering with integer priority
      \item \code{CK\_QUEUEING\_BFIFO} : FIFO ordering with bitvector priority
      \item \code{CK\_QUEUEING\_BLIFO} : LIFO ordering with bitvector priority
      \item \code{CK\_QUEUEING\_LFIFO} : FIFO ordering with long integer priority
      \item \code{CK\_QUEUEING\_LLIFO} : FIFO ordering with long integer priority
    \end{itemize}
 \end{itemize}
\end{frame}

\begin{frame}[fragile]
  \frametitle{Changing Message Order}
  \framesubtitle{Storage of Priority}
  \begin{itemize}
    \item If using integer/long integer/bit vector priorities, one needs to reserve memory in messages to store priority
     \begin{itemize}
       \item Last size argument (in bits) in brackets to new
     \end{itemize}
    \begin{lstlisting}
      MyVarsizeMsg *msg = new (10,7, 8*sizeof(int))         
                                      MyVarsizeMsg(<constructor args>);

      *(int*)CkPriorityPtr(msg) = prio; //set priority

      int * prioMsg = CkPriorityPtr(msg)  //get priority
    \end{lstlisting}
    \item LIFO/FIFO used to break ties if multiple messages have same priority
  \end{itemize}
\end{frame}

\begin{frame}[fragile]
  \frametitle{Changing Message Order}
  \framesubtitle{Marshalled Messages}
  \begin{itemize}
    \item queueingtype can be set for CkEntryOptions , which is passed to an entry method invocation as the optional last parameter
    \begin{lstlisting}
      CkEntryOptions opts1, opts2;
      opts1.setQueueing(CK_QUEUEING_FIFO);
      opts2.setQueueing(CK_QUEUEING_LIFO);
      chare.entry_name(arg1, arg2, opts1);
      chare.entry_name(arg1, arg2, opts2);
    \end{lstlisting}
    \item \code{opts.setPriority(prio\_t integerPrio);} set integer priorities
    \item \code{opts.setPriority(int prioBits,const prio\_t *prioPtr)}  set bit vector priority using prioPtr
  \end{itemize}
\end{frame}

\begin{frame}[fragile]
  \frametitle{Customizing Message Handling}
  \begin{itemize}
    \item By default, Charm generates following three methods for each Message class
    \begin{itemize}
      \item \code{static void* alloc(int msgnum, size\_t size, int* array, int priobits);}
      \item \code{static void* pack(mtype*);}
      \item \code{static mtype* unpack(void*);}
    \end{itemize}
   \item One may override these with their own implementation
   \begin{itemize}
     \item Useful for non-contiguous allocation
     \item \code{alloc} is invoked when \code{new} is called
     \item \code{pack} and \code{unpack} are for sending/receiving the message, and should be according to alloc
     \item Optimized \code{pack}/\code{unpack} using application knowledge
   \end{itemize}
  \end{itemize}
\end{frame}

\begin{frame}[fragile]
  \frametitle{Custom-packed Messages}
  \begin{itemize}
    \item How to pack messages that contain pointers to structures such as trees or graphs?
    \begin{itemize}
       \item i.e. pointers leading to pointers,
       \item Manual section: 10.1.3.1
    \end{itemize}
  \end{itemize}
\end{frame}

\begin{frame}[fragile]
  \frametitle{Messages}
  \framesubtitle{Motivation Summary}
  \begin{itemize}
    \item Avoids extra copy
    \item Can be custom packed
    \item Reusable
    \item Useful for transfer of complex data structures
  \end{itemize}
\end{frame}

\begin{frame}[fragile]
  \frametitle{Groups}
  \begin{itemize}
    \item PE bound chare-array with one chare per PE
    \item In .ci file, 
    \begin{lstlisting}
      group ExampleGroup  {
       // Interface specifications as for normal chares
       // For instance, the constructor ...
       entry ExampleGroup(parameters1);
       // ... and an entry method
       entry void someEntryMethod(parameters2);
       };
    \end{lstlisting}
    \item No difference in .h and .C file definitions
  \end{itemize}
\end{frame}

\begin{frame}[fragile]
  \frametitle{Groups}
  \framesubtitle{Example use}
  \begin{itemize}
    \item  One can obtain pointer to the local group member using groupProxy.ckLocalBranch()
    \begin{itemize}
      \item Access the local member as a regular C++ object
    \end{itemize}
   \item Consider a case:
   \begin{itemize}
      \item each chare on a processor performs a task and wants to pass some information to the mainchare
      \begin{itemize}
         \item Assume reduction is not feasible
      \end{itemize}
      \item One way is to directly send this information to mainchare from every chare (too many messages)
      \item Alternatively, each chare may deposit the information to the local branch of a group
      \begin{itemize}
        \item How to obtain a pointer to local branch? Manual 8.1.3
        \item \code {Gclass * g = grpProxy.ckLocalBranch();}
        \item \code{g} is a regular pointer to a C++ object
        \item Group member can pass this information to the mainchare after optional pre-processing when all chares have deposited 
      \end{itemize}
    \end{itemize}
  \end{itemize}
\end{frame}

\begin{frame}[fragile]
  \frametitle{Node Groups}
  \begin{itemize}
    \item A chare-array with one chare per node
    \begin{itemize}
      \item In non-smp mode groups and node-groups are same
    \end{itemize}
    \item No difference in .h and .C
    \item Creation and usage same as others
    \item An entry method on a node-group member may be executed on any PE of the node
    \item Concurrent execution of two entry methods of a node-group member may happen
    \begin{itemize}
      \item Use \code{[exclusive]} for entry methods which are unsuitable for reentrance safety
    \end{itemize}
  \end{itemize}
\end{frame}

\begin{frame}[fragile]
  \frametitle{Customizing Entry Methods}
  \begin{itemize}
    \item \code{threaded} – executed using separate thread
    \begin{itemize}
      \item each thread has a stack, and may be suspended
      \item to set stack’s size use +stacksize <size in bytes> 
    \end{itemize}
    \item \code{inline} – entry method invoked immediately if destination chare on same PE
    \item \code{sync} - returns a value
    \begin{itemize}
      \item blocking call
    \end{itemize}
    \item \code{reductiontarget} – target of an array reduction
    \begin{itemize}
      \item Takes parameter marshaled arguments
    \end{itemize}
    \item \code{notrace} – not traced for projections
  \end{itemize}
\end{frame}

\begin{frame}[fragile]
  \frametitle{Customizing Entry Methods}
  \begin{itemize}
    \item \code{expedited} – entry method skips the priority-based message queue in Charm++ runtime (for groups)
    \item \code{immediate} - skips the message scheduling queue (for any chare array)
    \item \code{nokeep} – message belongs to Charm
    \item \code{exclusive} – mutual exclusion on execution of entry methods on node-groups 
    \item \code{python} – can be called from python scripts
  \end{itemize}
\end{frame}

