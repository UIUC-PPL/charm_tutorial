\documentclass{beamer}

\usepackage{fancyvrb}
\usepackage{color, colortbl}
\usepackage{listings}
\usepackage{url}
\usepackage{array}
\usepackage{calc}
\usepackage{ctable}
\usepackage{amsmath}
\usepackage{cite}
\usepackage{graphicx}
\usepackage{listings}
\usepackage{xspace}
\usepackage{hyperref}
\usepackage{subfigure}
\usepackage{multicol}
\definecolor{lightgray}{rgb}{0.9,0.9,0.9}



\lstset{ %
language=C++,                % choose the language of the code
basicstyle=\tiny,       % the size of the fonts that are used for the code
%numbers=left,                   % where to put the line-numbers
numbers=none,                   % where to put the line-numbers
numberstyle=\scriptsize,      % the size of the fonts that are used for the line-numbers
stepnumber=1,                   % the step between two line-numbers. If it's 1 each line will be numbered
numbersep=15pt,                  % how far the line-numbers are from the code
backgroundcolor=\color{lightgray},  % choose the background color. You must add \usepackage{color}
%backgroundcolor=none,  % choose the background color. You must add \usepackage{color}
showspaces=false,               % show spaces adding particular underscores
showstringspaces=false,         % underline spaces within strings
showtabs=false,                 % show tabs within strings adding particular underscores
frame=single,	                % adds a frame around the code
%frame=none,	                % adds a frame around the code
tabsize=2,	                % sets default tabsize to 2 spaces
%captionpos=b,                   % sets the caption-position to bottom
captionpos=n,
%basicstyle=\small,
%basicstyle=\small\sffamily,
basicstyle=\sffamily\small,
%basicstyle=\ttfamily\small,
breaklines=true,                % sets automatic line breaking
breakatwhitespace=false,        % sets if automatic breaks should only happen at whitespace
columns=fullflexible,
title=\lstname,                 % show the filename of files included with \lstinputlisting; also try caption instead of title
escapeinside={\%*}{*)},          % if you want to add a comment within your code
morekeywords={chare,mainchare,module,mainmodule,entry,readonly,array,serial,for,when,if,then,else,overlap,while,forall,threaded,sync,message},
aboveskip=2pt,
belowskip=2pt,
lineskip=0pt,
xleftmargin=1em,
xrightmargin=1em,
%xleftmargin=10pt
abovecaptionskip=0pt,
belowcaptionskip=0pt,
}

\hypersetup{
    colorlinks,%
    citecolor=black,%
    filecolor=black,%
    linkcolor=black,%
    urlcolor=magenta
}


\usefonttheme{professionalfonts}
\usetheme{Boadilla}
\usecolortheme{beaver}

%\AtBeginSubsection[]
%{
%    \begin{frame}{Outline}
%        \tableofcontents[currentsection,currentsubsection]
%    \end{frame}
%}

\AtBeginSection[]{
%  \setbeamercolor{section in toc shaded}{use=structure,fg=structure.fg}
%  \setbeamercolor{section in toc}{fg=mycolor}
%  \setbeamercolor{subsection in toc shaded}{fg=black}
%  \setbeamercolor{subsection in toc}{fg=mycolor}
  \frame<beamer>{\begin{multicols}{2}
  \frametitle{Outline}
  \setcounter{tocdepth}{2}  
%  \tableofcontents[currentsection,subsections]
  \tableofcontents[currentsection,currentsubsection]
\end{multicols} 
 }
}


\newcommand{\charm}{Charm++}
\newcommand{\code}[1]{\colorbox{lightgray}{\texttt{#1}}}
\newcommand{\transition}[1]{\begin{frame}[plain]\begin{center}\LARGE #1\end{center}\end{frame}}
\newcommand{\comment}[1]{ }
\newcommand{\eat}[1]{ }

\DefineVerbatimEnvironment{codeverb}{Verbatim}{fontsize=\small}

\let \isForClass 1
\if \isForClass 1
  \newcommand{\removeForClass}[2]{#2}
  \else
  \newcommand{\removeForClass}[2]{#1}
\fi


\title[Parallel Migratable Objects]{Programming with Parallel Migratable Objects}
\institute[UIUC]{Parallel Programming Laboratory \\University of Illinois Urbana-Champaign}
\author{Laxmikant V.~Kal\'e, Eric Bohm}
\date{\today}

\begin{document}

\frame{\titlepage}

\transition{Please check http://charm.cs.illinois.edu/tutorial/ for the latest material}
%\tableofcontents

\section[Concepts]{Introduction}
\begin{frame}[t]
\frametitle{Harnessing Parallelism: Challenges}
\framesubtitle{Trends in System Architecture}
    \begin{itemize}
        \item Frequencies have stopped increasing
        \item Memory costs are high
          \begin{itemize}
          \item Relatively low per core memory
          \end{itemize}
        \item Increasing heterogeneity
          \begin{itemize}
          \item Accelerators, SMT
          \end{itemize}
        \item Energy, power, and thermal considerations
        \item Frequent component failures
     \end{itemize}
\end{frame}

\begin{frame}[shrink]
\frametitle{Harnessing Parallelism: Challenges}
\framesubtitle{Trends in System Architecture}
  \begin{columns}
    \begin{column}{0.60\textwidth}
      \begin{itemize}
      \item However, compute resources are not faster cores, but \textbf{more cores}
      \item Unprecedented levels of available concurrency
        \begin{itemize}
        \item IBM BG/Q
          \begin{itemize}
          \item `Sequoia': 1,572,864 cores
          \item `Mira': 786,432 cores
          \end{itemize}
        \item Cray
          \begin{itemize}
          \item XE6 `Bluewaters`: $>$ 380,000 cores
          \item XK6 `Titan': 299,008 cores
          \end{itemize}
        \item K Supercomputer: 705,024 cores
        \end{itemize}
      \end{itemize}
    \end{column}
    \begin{column}{0.40\textwidth}
      \includegraphics[width=1\textwidth]{figures/mira.jpg}
    \end{column}
  \end{columns}

  \begin{itemize}
    \item Mid-size clusters will be ubiquitous 
  \end{itemize}

    \pause
    \begin{block}{Implications}
        \begin{itemize}
            \item Each thread of execution has to:
                \begin{itemize}
                    \item operate on lesser data
                    \item wait relatively longer for remote data
                \end{itemize}
            \item Have to operate in \textbf{strong scaling} regime
            % Even if you don't do anything, an 8GB problem will have to run on
            % more cores a few years from now simply because there will be many
            % more cores for the same 8GB 
            % If a product has to stay ahead of the competition, it has to scale
            % the same problem to even more cores with time to run faster
        \end{itemize}
    \end{block}
\end{frame}

% \begin{frame}[t]
% \frametitle{Harnessing Parallelism: Challenges}
% \framesubtitle{Trends in System Architecture}
%     \begin{itemize}
%         \item Growing heterogeneity: Each thread of execution has different abilities
%             \begin{itemize}
%                 \item NUMA cores on each node
%                 \item Secondary threads in SMT can be less capable (IBM Power 7)
%                 \item Accelerators (NVIDIA / AMD GPUs, Intel MIC etc)
%                 \item Some cores run system daemons
%                 \item Combinations of above
%                     \begin{itemize}
%                         \item GPUs only on some nodes (BlueWaters: NCSA)
%                         \item MIC + SMT nodes (Stampede: TACC)
%                     \end{itemize}
%             \end{itemize}
%         \pause
%         \item Each architecture expects different treatment
%             \begin{itemize}
%                 \item GPUs expect evenly divided pieces
%                 \item SMT might expect uneven pieces (IBM Power7)
%             \end{itemize}
%     \end{itemize}
%     \pause
%     \begin{block}{Implications}
%         \begin{itemize}
%             \item Achieving load balance on such hardware is challenging
%             \item Minimizing idle time is extremely difficult
%         \end{itemize}
%     \end{block}
% \end{frame}


% \begin{frame}[t]
% \frametitle{Harnessing Parallelism: Challenges}
% \framesubtitle{Trends in System Architecture}
%   \begin{itemize}
%   \item Extracting performance on tighter power / energy budget
%   \item Hardware component failures / faults
%   \end{itemize}
% \end{frame}

% \begin{frame}[fragile]
% \frametitle{Harnessing Parallelism: Challenges}
% \framesubtitle{Application characteristics}
%   \begin{itemize}
%   \item Complex physics in multiple, interacting modules
%   \item Adaptive, spatial and temporal resolutions
%   \item Need for faster solutions (not just larger problems)
%   \end{itemize}
% \end{frame}

\begin{frame}[t]
\frametitle{Harnessing Parallelism: Challenges}
%\frametitle{Observations}
\framesubtitle{Next-generation Applications}
  \begin{itemize}
    \item Need for strong scaling
      \begin{itemize}
      \item faster solutions (not just larger problems)
      \end{itemize}
    \item Application Characteristics
    \begin{itemize}
      \item Multi-resolution
        \begin{itemize}
      \item Adaptive, spatial and temporal resolutions
      \item Dynamic/adaptive refinements: to handle application
        variation
        \end{itemize}
      \item Multi-module (multi-physics)
        \begin{itemize}
        \item Complex physics in multiple, interacting modules
        \end{itemize}
      \item Adapt to a volatile computational environment
      \item Exploit heterogeneous architecture
      \item Deal with thermal and energy considerations
    \end{itemize}
    \pause
    \item So? Consequences:
    \begin{itemize}
      \item Must support automated resource management
      \item Must support interoperability and parallel composition
    \end{itemize}
  \end{itemize}
\end{frame}

\begin{frame}[t]
\frametitle{Harnessing Parallelism: Challenges}
\framesubtitle{Programming Models: MPI}
  \begin{itemize}
    \item Highly successful
    \item Universally used
    \item Has supported application evolution from gigascale to petascale
  \end{itemize}
  \begin{itemize}
    \item Library
    \item Communication primitives
  \end{itemize}
  \begin{itemize}
    \item MPI does not directly support automated resource management
      (e.g. load balancing, fault tolerance, etc.)
  \end{itemize}
\end{frame}

% \begin{frame}[fragile]
% \frametitle{Harnessing Parallelism: Challenges}
% \framesubtitle{2D Jacobi Iterations: 5-point Stencil}
%    \begin{center} \includegraphics[width=0.6\textwidth]{figures/2DJacobi_NeighborComm.jpg} \end{center}
% \end{frame}


% \begin{frame}[fragile]
% \frametitle{Harnessing Parallelism: Challenges}
% \framesubtitle{2D Jacobi Iterations: No Overlap}
%     \lstinputlisting{code/jacobi2D-mpi.c}
% \end{frame}


% \begin{frame}[fragile, shrink]
% \frametitle{Harnessing Parallelism: Challenges}
% \framesubtitle{2D Jacobi Iterations: Some Overlap}
%     \lstinputlisting{code/jacobi2D-mpi-nonblocking.c}
% \end{frame}

\comment{
\begin{frame}[fragile]
\frametitle{Harnessing Parallelism: Challenges}
\framesubtitle{Composing Independent Parallel Modules}
  \frametitle{Harnessing Parallelism}
  \framesubtitle{Composing Independent Parallel Modules}
  \begin{itemize}
    \item Sequential blocks (SEBs) in two modules can be partitioned in space
  \end{itemize}
  \begin{center}
    \includegraphics[width=0.8\textwidth]{figures/spaceDivision.pdf}
  \end{center}
\end{frame}

\begin{frame}[fragile]
\frametitle{Harnessing Parallelism: Challenges}
\framesubtitle{Composing Independent Parallel Modules}
  \begin{itemize}
    \item Sequential blocks (SEBs) in two modules can be partitioned in time
  \end{itemize}
  \begin{center}
    \includegraphics[width=0.8\textwidth]{figures/timeDivision.pdf}
  \end{center}
\end{frame}

\begin{frame}[fragile]
\frametitle{Stuff you already know}
\framesubtitle{Some issues with procedural code}

  \begin{columns}
    \begin{column}{0.4\textwidth}
      \lstinputlisting[basicstyle=\tiny]{code/jacobi2D-mpi.c}
    \end{column}
    \begin{column}{0.6\textwidth}
      \begin{itemize}
      \item Very similar to sequential code
      \item Tied to notion of communicating ranks
      \item Explicitly states order of computation and communication
      \item However no semantic connection between computation and its communication dependencies
      \end{itemize}
    \end{column}
  \end{columns}

\end{frame}
}


\subsection[Object Design]{Object Design}
\transition{Charm++ builds upon a proven approach: objects}
  \begin{frame}[fragile]
\frametitle{Stuff you already know}
\framesubtitle{Some issues with procedural code}

  \begin{columns}
    \begin{column}{0.4\textwidth}
      \lstinputlisting[basicstyle=\tiny]{code/jacobi2D-mpi.c}
    \end{column}
    \begin{column}{0.6\textwidth}
      \begin{itemize}
      \item Very similar to sequential code
      \item Tied to notion of communicating ranks
      \item Explicitly states order of computation and communication
      \item However no semantic connection between computation and its communication dependencies
      \item Data is not well encapsulated.
      \end{itemize}
    \end{column}
  \end{columns}

\end{frame}


\begin{frame}[shrink]
\frametitle{Stuff you already know}
\framesubtitle{Benefits of Object-based code}
    \begin{itemize}
        \item Objects encapsulate data
        \item Methods represent functionality relevant to that data
        \item Method invocations can modify / update state of the object / data
        \item Computation can be expressed in terms of objects interacting via method invocations
    \end{itemize}
    \begin{block}{}
    \begin{itemize}
        \item Methods are natural units of sequential computation on object data
        \item Thoughtful design yields focused methods with single purpose
        \item Naturally expresses an object's response to inputs (signals / data)
    \end{itemize}
    \end{block}
    \begin{itemize}
        \item Nothing new
        \item Still quite uncommon in HPC code
        \item Its not about language syntax. Its about program structure
    \end{itemize}
}

%\transition{A proven approach: method-driven objects}
\begin{frame}[fragile]
  \frametitle{Charm++}
  \begin{itemize}
    \item C++-based parallel runtime system
      \begin{itemize}
      \item Composed of a set of globally-visible parallel objects that interact
      \item The objects interact by asynchronously invoking methods on each other
      \end{itemize}
    \item Charm++ runtime
      \begin{itemize}
      \item Manages the parallel objects and (re)maps them to processes
      \item Provides scheduling, load balancing, and a host of other features,
        requiring little user intervention
      \end{itemize}
  \end{itemize}
\end{frame}

\begin{frame}[fragile]
  \frametitle{Globally-Visible Objects}
  \begin{center}
    \includegraphics[width=0.8\textwidth]{figures/objectGlobalAddress.pdf}
  \end{center}
  \begin{itemize}
    \item Certain ``special'' object \emph{instances} are:
      \begin{itemize}
      \item first-class citizens in the parallel address space,
      \item with unique location-independent names
      \end{itemize}
    \item Under the hood, the runtime handles locality and provides the
      mechanisms to promote objects to the parallel space
  \end{itemize}
\end{frame}

\begin{frame}[fragile]
  \frametitle{Globally-Visible Methods}
  \begin{center}
    \includegraphics[width=\textwidth]{figures/objectMethodGlobalAddress.pdf}
  \end{center}
  \begin{itemize}
    \item How can objects communicate across address spaces?
      \begin{itemize}
      \item Just like a sequential object-oriented language, an object's
        reference is used to invoke a method
      \item In the parallel space, this is a handle that is location
          transparent
      \item A method invocation becomes an act of communication
      \end{itemize}
  \end{itemize}
\end{frame}

\begin{frame}[fragile]
  \frametitle{Method-Driven Asynchronous Communication}
  \begin{center}
    \includegraphics[width=\textwidth]{figures/objectSequence.pdf}
  \end{center}
  \begin{itemize}
  \item What happens if an object waits for a return value from a method
    invocation?
    \begin{itemize}
    \item Performance
    \item Latency
    \item Reasoning about correctness
    \end{itemize}
  \end{itemize}
\end{frame}

\begin{frame}[fragile]
  \frametitle{Design Principle: Do not wait for remote completion}
  \begin{center}
    \includegraphics[width=\textwidth]{figures/objectSequenceAsync.pdf}
  \end{center}
  \begin{itemize}
  \item Hence, method invocations should be asynchronous
    \begin{itemize}
    \item No return values
    \end{itemize}
  \item Computations are driven by the incoming data
    \begin{itemize}
    \item Initiated by the sender or method caller
    \end{itemize}
  \end{itemize}
\end{frame}

% \begin{frame}[fragile]
%   \frametitle{Dependencies and Dataflow}
%     \begin{columns}
%     \begin{column}{0.55\textwidth}
%       \begin{itemize}
%       \item A method invocation expresses a dependency in the computation
%       \item For example, we can now express a LU decomposition in its natural
%         form
%         % \begin{itemize}
%         % \item A directed acyclic graph
%         % \end{itemize}
%       \item More sophisticated dependencies can be expressed with a scripting
%         language in Charm++
%       \end{itemize}
%     \end{column}
%     \begin{column}{0.45\textwidth}
%       \includegraphics[width=0.7\textwidth]{figures/ludag.png}
%     \end{column}
%   \end{columns}
% \end{frame}

\subsection[Execution Mode]{Execution Model}
\begin{frame}
\frametitle{import the pptx slides from executionModelandBenefits.pptx}
\end{frame}

\section[Hello World]{Hello World}
\begin{frame}[fragile]
   \frametitle{Hello World Example}\scriptsize
\begin{itemize}
\item \texttt{hello.ci} file
   \begin{lstlisting}
mainmodule hello {
  mainchare Main {
    entry Main(CkArgMsg *m);
  };
};
   \end{lstlisting}
\item \texttt{hello.cpp} file
\lstset{basicstyle=\footnotesize}
   \begin{lstlisting}
#include <stdio.h>
#include "hello.decl.h"

class Main : public CBase_Main {
  public: Main(CkArgMsg* m) {
    ckout << "Hello World!" << endl;
    CkExit();
  };
};

#include "hello.def.h"
   \end{lstlisting}
\end{itemize}
\end{frame}

\begin{frame}[fragile]
   \frametitle{Hello World with Chares}\scriptsize
\begin{columns}
\begin{column}{.45\linewidth}
\texttt{hello.ci} file
   \begin{lstlisting}
mainmodule hello {
  mainchare Main { 
   entry Main(CkArgMsg *m); 
  };
  chare Singleton {
   entry Singleton();
  };
};
   \end{lstlisting}
\end{column}
\begin{column}{.55\linewidth}
\texttt{hello.cpp} file
\begin{lstlisting}
#include <stdio.h>
#include "hello.decl.h"

class Main : public CBase_Main {
  public: Main(CkArgMsg* m) {
    CProxy_Singleton::ckNew();
  };
};

class Singleton : public CBase_Singleton {
  public: Singleton() {
    ckout << "Hello World!" << endl;
    CkExit();
  };
};
#include "hello.def.h"
\end{lstlisting}
\end{column}
\end{columns}
\end{frame}


\section[Benefits]{Benefits of Charm++}
\begin{frame}[fragile]
\includegraphics[width=0.9\textwidth]{figures/charmOutline.png}
\end{frame}

\begin{frame}[fragile]
\frametitle{Load Balancing}
\begin{itemize}
 \item Static
   \begin{itemize}
   \item Irregular applications
   \item Programmer shouldn't have to figure out ideal mapping
   \end{itemize}
 \item Dynamic
   \begin{itemize}
   \item Applications are increasingly using adaptive strategies
   \item Abrupt refinements
   \item Continuous migration of work: e.g. particles in MD
   \end{itemize}
 \item Challenges
   \begin{itemize}
   \item Performance limited by most overloaded processor
   \item The chance that one processor is severely overloaded gets higher as
     \#processors increases
   \end{itemize}
\end{itemize}
\begin{center}\textbf{Migratable Objects Empower Automated Load Balancing!}\end{center}
\end{frame}


\begin{frame}[fragile]
\frametitle{Principle of Persistence}
\begin{itemize}
 \item Once the computation is expressed in terms of its natural (migratable)
   objects
 \item Computational loads and communication patterns \emph{tend to} persist,
   even in dynamic computations
 \item So, recent past is a good predictor of near future
 \item The runtime system mediates communication between objects, and schedules
   execution of objects, so it can introspect: record computation loads and
   communication graphs
\end{itemize}
\end{frame}


\begin{frame}[fragile]
\frametitle{A quick Example}
\framesubtitle{Weather Forecasting in BRAMS}
\begin{itemize}
 \item Brams: Brazillian weather code (based on RAMS)
 \item AMPI version (Eduardo Rodrigues, with Mendes and J. Panetta)
\end{itemize}
\includegraphics[width=0.9\textwidth]{figures/bramsVisual.png}
\end{frame}


\begin{frame}[fragile]
\frametitle{Basic Virtualization of BRAMS}
\includegraphics[width=0.5\textwidth]{figures/bramsNonVirtual.png}%
\includegraphics[width=0.5\textwidth]{figures/bramsVirtual.png}
\end{frame}

\begin{frame}[fragile]
\frametitle{Baseline: 64 objects on 64 processors}
\begin{center}\includegraphics[width=0.9\textwidth]{figures/usageNonVirtual.png}\end{center}
\end{frame}

\begin{frame}[fragile]
\frametitle{Over-decomposition: 1024 objects on 64 processors}
\framesubtitle{Benefits from communication/computation overlap}
\begin{center}\includegraphics[width=0.9\textwidth]{figures/usageVirtual.png}\end{center}
\end{frame}

\begin{frame}[fragile]
\frametitle{With Load Balancing: 1024 objects on 64 processors}
\begin{center}
\begin{itemize}
\item No overdecomp (64 threads): 4988 sec
\item Overdecomp into 1024 threads: 3713 sec
\item Load balancing (1024 threads): 3367 sec
\end{itemize}
\includegraphics[width=0.8\textwidth]{figures/usageLB.png}
\end{center}
\end{frame}

\section[Charm++]{Charm++ Syntax}
\begin{frame}
  \frametitle{Charm++ File structure}
  \begin{itemize}
    \item C++ objects (including Charm++ objects)
      \begin{itemize}
      \item Defined in regular \texttt{.h} and \texttt{.cpp} files
      \end{itemize}
    \item Chare objects, entry methods (asynchronous methods)
      \begin{itemize}
      \item Defined in \texttt{.ci} file
      \item Implemented in the \texttt{.cpp} file
      \end{itemize}
  \end{itemize}
  \begin{center}
    \includegraphics[width=0.6\textwidth]{figures/charmFiles.png}
  \end{center}
\end{frame}

\begin{frame}[fragile]
  \frametitle{Charm Interface: Modules}
  \begin{itemize}
    \item Charm++ programs are organized as a collection of modules
    \item Each module has one or more chares
    \item The module that contains the \textit{mainchare}, is declared as the
      \texttt{mainmodule}
    \item Each module, when compiled, generates two files:
      \code{<modulename>.decl.h} and \code{<modulename>.def.h}
  \end{itemize}
  \begin{center}
  \begin{lstlisting}
    [main]module <modulename> {
      //... chare definitions ...
    };
  \end{lstlisting}
  \end{center}
\end{frame}

\begin{frame}[fragile]
  \frametitle{Charm Interface: Chares}
  \begin{itemize}
    \item Chares are parallel objects that are managed by the RTS
    \item Each chare has a set \textit{entry methods}, which are asynchronous
      methods that may be invoked remotely
    \item The following code, when compiled, generates a C++ class
      \code{CBase\_<charename>} that encapsulates the RTS object
    \item This generated class is extended and implemented in the \texttt{.cpp}
      file
  \end{itemize}
  \begin{lstlisting}
    [main]chare <charename> {
      //... entry method definitions ...
    };

    class MyChare : public CBase_<charename> {
      //... entry method implementations ...
    };
  \end{lstlisting}
\end{frame}

\begin{frame}[fragile]
  \frametitle{Charm Interface: Entry Methods}
  \begin{itemize}
  \item Entry methods are C++ methods that can be remotely and asynchronously
    invoked by another chare
  \end{itemize}
  \begin{lstlisting}
    entry <charename>(); /* constructor entry method */
    entry void foo();
    entry void bar(int param);

    <charename>::<charename>() { /*... constructor code ...*/ }

    <charename>::foo() { /*... code to execute ...*/ }

    <charename>::bar(int param) { /*... code to execute ...*/ }
  \end{lstlisting}
\end{frame}

\begin{frame}[fragile]
   \frametitle{Charm Interface: \texttt{mainchare}}
   \begin{itemize}
     \item Execution begins with the \texttt{mainchare}'s constructor
     \item The \texttt{mainchare}'s constructor takes a pointer to
       system-defined class \code{CkArgMsg}
     \item \code{CkArgMsg} contains \code{argv} and \code{argc}
     \item The \texttt{mainchare} will often construct other parallel objects
       and then wait for them to finish
   \end{itemize}
\end{frame}

\begin{frame}[fragile]
   \frametitle{Charm Termination}
   \begin{itemize}
   \item There is a special system call \code{CkExit()} that terminates the
     parallel execution on all processors (but it is called on one processor)
     and performs the requisite cleanup
   \item The traditional \code{exit()} is insufficient because it only
     terminates one process, not the entire parallel job (and will cause a
     hang)
   \item \code{CkExit()} should be called when you can safely terminate the
     application (you may want to synchronize before calling this)
   \end{itemize}
\end{frame}

\begin{frame}
   \frametitle{Compiling a Charm++ Program}
   \begin{center}
     \includegraphics[width=0.9\textwidth]{figures/charmCompile.jpg}
   \end{center}
\end{frame}

\begin{frame}[fragile]
   \frametitle{Hello World Example}\scriptsize
\begin{itemize}
\item \texttt{hello.ci} file
   \begin{lstlisting}
mainmodule hello {
  mainchare Main {
    entry Main(CkArgMsg *m);
  };
};
   \end{lstlisting}
\item \texttt{hello.cpp} file
\lstset{basicstyle=\footnotesize}
   \begin{lstlisting}
#include <stdio.h>
#include "hello.decl.h"

struct Main : public CBase_Main {
  Main(CkArgMsg* m) {
    ckout << "Hello World!" << endl;
    CkExit();
  };
};

#include "hello.def.h"
   \end{lstlisting}
\end{itemize}
\end{frame}

\begin{frame}
  \frametitle{Hello World Example}
  \begin{itemize}
    \item Compiling
      \begin{itemize}
      \item \texttt{charmc hello.ci}
      \item \texttt{charmc -c hello.cpp}
      \item \texttt{charmc -o hello hello.cpp}
      \end{itemize}
    \item Running
      \begin{itemize}
      \item \texttt{./charmrun +p7 ./hello}
      \item The \texttt{+p7} tells the system to use seven cores
      \end{itemize}
    \end{itemize}
\end{frame}

\begin{frame}[fragile]
  \frametitle{Creating a Chare}
  \begin{itemize}
    \item A chare declared as \code{chare <charename> \{...\};} can be
      instantiated by the following call:
  \end{itemize}
\begin{lstlisting}
CProxy_<charename>::ckNew(... constructor arguments ...);
\end{lstlisting}
  \begin{itemize}
  \item To communicate with this class in the future, a \textit{proxy} to it
    must be retained
  \end{itemize}
\begin{lstlisting}
CProxy_<charename> proxy = 
  CProxy_<charename>::ckNew(... constructor arguments ...);
\end{lstlisting}
\end{frame}

\begin{frame}[fragile]
  \frametitle{Chare Creation Example: \texttt{.ci} file}
  \lstinputlisting{code/chareCreate.ci}
\end{frame}

\begin{frame}[fragile]
  \frametitle{Chare Creation Example: \texttt{.cpp} file}
  \lstinputlisting[basicstyle=\footnotesize]{code/chareCreate.C}
\end{frame}

\begin{frame}[fragile]
  \frametitle{Asynchronous Methods}
  \begin{itemize}
  \item Entry methods are invoked by performing a C++ method call on a chare's
    proxy
  \end{itemize}
  \begin{lstlisting}
CProxy_<charename> proxy =
  CProxy_<charename>::ckNew(... constructor arguments ...);

proxy.foo();
proxy.bar(5);
\end{lstlisting}
\begin{itemize}
\item The \code{foo} and \code{bar} methods will then be executed with the
  arguments, wherever \texttt{<charename>} happens to live
\item The policy is one-at-a-time scheduling (that is, one entry method on one
  chare executes on a processor at a time)
\end{itemize}
\end{frame}

\begin{frame}[fragile]
  \frametitle{Asynchronous Methods}
  \begin{itemize}
  \item Method invocation is not ordered (between chares, entry methods on one
    chare, etc.)!
  \item For example, if a chare executes this code:
  \begin{lstlisting}
CProxy_<charename> proxy = CProxy_<charename>::ckNew();
proxy.foo();
proxy.bar(5);
  \end{lstlisting}
  \item These prints may occur in \textbf{any} order
  \begin{lstlisting}
<charename>::foo() {
  ckout << "foo executes" << endl;
}

<charename>::bar(int param) {
  ckout << "bar executes with " << param << endl;
}
\end{lstlisting}
  \end{itemize}
\end{frame}

\begin{frame}[fragile]
  \frametitle{Asynchronous Methods}
  \begin{itemize}
  \item For example, if a chare invokes the same entry method twice:
  \begin{lstlisting}
proxy.bar(7);
proxy.bar(5);
  \end{lstlisting}%
  \item These may be delivered in \textbf{any} order
  \begin{lstlisting}
<charename>::bar(int param) {
  ckout << "bar executes with " << param << endl;
}
\end{lstlisting}
  \item Output
\begin{lstlisting}
bar executes with 5
bar executes with 7
\end{lstlisting}
\textbf{OR}
\begin{lstlisting}
bar executes with 7
bar executes with 5
\end{lstlisting}
  \end{itemize}
\end{frame}


\begin{frame}[fragile]
  \frametitle{Asynchronous Example: \texttt{.ci} file}
\begin{lstlisting}
mainmodule MyModule {
  mainchare Main {
    entry Main(CkArgMsg *m);
  };
  chare Simple {
    entry Simple(double y);
    entry void findArea(int radius, bool done);
  };
};
\end{lstlisting}
\end{frame}

\begin{frame}[fragile]
  \frametitle{Asynchronous Example: \texttt{.cpp} file}
\begin{itemize}
  \item Does this program execute correctly?
\end{itemize}
\scriptsize
\begin{lstlisting}[basicstyle=\tiny]
struct Main : public CBase_Main {
  Main(CkArgMsg* m) {
    double pi = 3.1415;
    CProxy_Simple sim =  CProxy_Simple::ckNew(pi);
    for (int i = 1; i< 10; i++) sim.findArea(i, false);
    sim.findArea(10, true);
  };
};

struct Simple : public CBase_Simple {
 float y;
 Simple(double pi) {
   y = pi;
   ckout << "Hello from a simple chare running on " << CkMyPe() << endl;
 }
 void findArea(int r, bool done) {
   ckout << "Area of a circle of radius" << r << " is " << y*r*r << endl;
   if (done) CkExit();
 }
};
\end{lstlisting}
\end{frame}

\begin{frame}[fragile]
  \frametitle{Data types and entry methods}
\begin{itemize}
  \item You can pass basic C++ types to entry methods (\texttt{int},
    \texttt{char}, \texttt{bool}, etc.)
  \item C++ STL data structures can be passed by including \texttt{pup\_stl.h}
  \item Arrays of basic data types can also be passed like this:\\
\begin{lstlisting}
entry void foobar(int length, int data[length]);

<charename>::foobar(int length, int* data) {
 // ... foobar code ...
}
\end{lstlisting}
\end{itemize}
\end{frame}

\begin{frame}[fragile]
  \frametitle{Chare Proxies}
  \begin{itemize}
  \item A chare's own proxy can be obtained through a special variable
    \code{thisProxy}
  \item Chare proxies can also be passed so chares can learn about others
  \item In this snippet, \code{<charename>} learns about a chare instance
    \code{main}, and then invokes a method on it:
  \end{itemize}
\begin{lstlisting}
entry void foobar2(CProxy_Main main);

<charename>::foobar2(CProxy_Main main) {
  main.foo();
}
\end{lstlisting}
\end{frame}

\begin{frame}[fragile]
  \frametitle{Chare Proxy Example}
\texttt{.ci} file
\begin{lstlisting}[basicstyle=\tiny]
  mainchare Main {
    entry Main(CkArgMsg *m);
    entry void finished();
  };
  chare Simple {
    entry Simple(CProxy_Main mainProxy);
  };
\end{lstlisting}
\texttt{.cpp} file
\begin{lstlisting}[basicstyle=\tiny]
  struct Main : public CBase_Main {
    Main(CkArgMsg *m) {
      CProxy_Simple::ckNew(thisProxy);      
    }
    void finished() { CkExit(); }
  };
  struct Simple : public CBase_Simple {
    Simple(CProxy_Main mainProxy) {
      ckout << "Hello from Simple" << endl;
      mainProxy.finished();
    }
  };
\end{lstlisting}
\end{frame}

\begin{frame}[fragile]
  \frametitle{Readonlys}
  \begin{itemize}
  \item A \textit{readonly} is a global (within a module) read-only variable
    that can only be written to in the \texttt{mainchare}'s constructor
  \item Can then be read (\textbf{not written!}) by any chare in the module
  \item It is declared in the \texttt{.ci} file:
\begin{lstlisting}
  readonly <type> <name>;
  readonly CProxy_Main mainProxy;
  readonly int numChares;
\end{lstlisting}
  \item And defined the the \texttt{.cpp} file:
\begin{lstlisting}
  <type> <name>;
  CProxy_Main mainProxy;
  int numChares;
\end{lstlisting}
  \item And set in the \texttt{mainchare}'s constructor
\begin{lstlisting}
  <charename>::<charename>(CkArgMsg *m) {
    mainProxy = thisProxy;
    numChares = 10;
  }
\end{lstlisting}


  \end{itemize}
\end{frame}

\begin{frame}[fragile]
  \frametitle{PI Example}
  \lstinputlisting{code/pi.ci}  
\end{frame}

\begin{frame}[fragile]
  \frametitle{PI Example}
  \lstinputlisting[basicstyle=\tiny]{code/piMaster.C}
\end{frame}

\begin{frame}[fragile]
  \frametitle{PI Example}
  \lstinputlisting{code/piWorker.C}
\end{frame}
\section[Task Parallelism]{Task Parallelism}
\begin{frame}[fragile]
  \frametitle{Task Parallelism with Objects}
  \begin{itemize}
    \item Divide-and-conquer
      \begin{itemize}
      \item Each object recursively creates $n$ objects that divide the problem
        into subproblems
      \item Each object $t$ then waits for all $n$ objects to finish and then may
        `combine' the responses
      \item At some point the recursion stops (at the bottom of the tree), and
        some sequential kernel is executed
      \item Then the result is propagated upward in the tree recursively
      \item Examples: fibonacci, quick sort, $\ldots$
      \end{itemize}
  \end{itemize}
\end{frame}

\begin{frame}[fragile]
  \frametitle{Fibonacci Example}
  \begin{itemize}
    \item Each \code{Fib} object is a task that performs one of two actions:
      \begin{itemize}
        \item Creates two new \code{Fib} objects to compute $fib(n-1)$ and
          $fib(n-2)$ and then waits for the response, adding up the two
          responses when they arrive
          \begin{itemize}
          \item After both arrive, sends a response message with the result to
            the parent object
          \item Or prints the value and exits if it is the root
          \end{itemize}
        \item If $n = 1$ or $n = 0$ (passed down from the parent) it sends a
          response message with $n$ back to the parent object
      \end{itemize}
  \end{itemize}
\end{frame}

\begin{frame}[fragile]
  \frametitle{Fibonacci Example}
  \lstinputlisting[basicstyle=\tiny]{code/fibPsuedo.cpp}
\end{frame}

\begin{frame}[fragile]
  \frametitle{Fibonacci Execution}
   \begin{center}
     \includegraphics<1>[width=0.9\textwidth]{figures/tree1.pdf}
     \includegraphics<2>[width=0.9\textwidth]{figures/tree2.pdf}
     \includegraphics<3>[width=0.9\textwidth]{figures/tree3.pdf}
     \includegraphics<4>[width=0.9\textwidth]{figures/tree4.pdf}
     \includegraphics<5>[width=0.9\textwidth]{figures/tree5.pdf}
     \includegraphics<6>[width=0.9\textwidth]{figures/tree6.pdf}
     \includegraphics<7>[width=0.9\textwidth]{figures/tree7.pdf}
     \includegraphics<8>[width=0.9\textwidth]{figures/tree8.pdf}
     \includegraphics<9>[width=0.9\textwidth]{figures/tree9.pdf}
     \includegraphics<10>[width=0.9\textwidth]{figures/tree10.pdf}
   \end{center}
\end{frame}

\transition{Overdecomposing Your Application}
\begin{frame}
  \frametitle{Object-based Over-decomposition}
  \begin{itemize} 
    \item Let the programmer decompose computation into objects
    \begin{itemize}
       \item Work units, data-units, composites
    \end{itemize}
    \item Let an intelligent runtime system assign objects to processors
    \begin{itemize}
      \item RTS can change this assignment (mapping) during execution
      \item Locality of data references is a critical attribute for performance 
      \item A parallel object can access only its own data
      \item Asynchronous method invocation for accessing other objects’ data
      \item RTS can schedule work whose dependencies have been satisfied
    \end{itemize}
\end{itemize}
\end{frame}

% \begin{frame}
%  \frametitle{Object Boundaries}
%     \begin{itemize}
%       \item Parameters to methods of objects represent data dependencies
%       \item Programmer designs objects and methods with explicit dependencies
%       \item Flow of control is then the chain of dependencies
%       \item Overlap of independent computations happens automatically as dependencies are met via message delivery
%     \end{itemize}
%  \end{frame}

%% Grainsize slides
\begin{frame}
  \frametitle{Amdahl’s Law and Grainsize}
  \begin{itemize}
    \item Original ``law'':
      \begin{itemize}
      \item If a program has $K$\% sequential section, then speedup is limited
        to $\frac{100}{K}$.
        \begin{itemize}
        \item If the rest of the program is parallelized completely
        \end{itemize}
      \end{itemize}
    \item Grainsize corollary:
      \begin{itemize}
      \item If any individual piece of work is $> K$ time units, and the
        sequential program takes $T_{seq}$, 
        \begin{itemize}
        \item Speedup is limited to $\frac{T_{seq}}{K}$
        \end{itemize}
      \end{itemize}
    \item So:
      \begin{itemize}
      \item Examine performance data via histograms to find the sizes of
        remappable work units
      \item If some are too big, change the decomposition method to make
        smaller units
      \end{itemize}
  \end{itemize}
\end{frame}

\begin{frame}
  \frametitle{Overdecomposition and Grainsize}
  \begin{itemize}
    \item Common misconception: overdecomposition must be expensive
    \item (working) Definition: the amount of computation per potentially
      parallel event (task creation, enqueue/dequeue, messaging,
      locking, etc.)
  \end{itemize}
%   \begin{center} \includegraphics[width=0.7\textwidth]{figures/grain1.png} \end{center}
\end{frame}

\begin{frame}
  \frametitle{Grainsize and Overhead}
  \begin{itemize}
    \item What is the ideal grainsize?
    \item Should it depend on the number of processors?
    
  \end{itemize}
  \begin{center}
    $T_1 = T \left( 1 + \frac{v}{g} \right)$\\
    $T_p = max \left\{ g, \frac{T_1}{p} \right\}$\\
    $T_p = max \left\{ g, \frac{T\left( 1+ \frac{v}{g} \right)}{p} \right\}$\\
    $v$: overhead per message,\\
    $T_p$: $p$ processor completion time\\
    $g$: grainsize (computation per message)
  \end{center}
\end{frame}

\begin{frame}
  \frametitle{Grainsize and Scalability}
  \begin{center} \includegraphics[width=0.7\textwidth]{figures/grain2.png} \end{center}
\end{frame}

\begin{frame}
  \frametitle{Grainsize Study for Jacobi3D}
  \begin{center}
  \includegraphics[width=0.7\textwidth]{figures/jacobi-grainsize-halfmemory.pdf} \end{center}
\end{frame}

\begin{frame}
  \frametitle{Grainsize Study for Stencil Computation}
  \begin{itemize}
    \item Blue Waters (JYC) , 2 nodes, 32 cores each
  \end{itemize}
  \begin{center} \includegraphics[width=0.7\textwidth]{figures/jacobi-grainsize.pdf} \end{center}

Typically, having tens of chares per code is adequate (although
reasoning should be based on computation per message)

\end{frame}

\begin{frame}[fragile]
\frametitle{Grainsize and Load Balancing}
\begin{itemize}
\item[] How Much Balance Is Possible?
\end{itemize}
\begin{columns}
  \begin{column}[T]{2.8in}
  \includegraphics[width=2.8in, height=2.0in]{figures/histogramGrains}
  \end{column}

  \begin{column}[T]{5cm}
  Solution:\\ 
  Split compute objects that may have too much work,
  using a heuristic based on number of interacting atoms
  \end{column}
\end{columns}
\end{frame}

\begin{frame}[fragile]
\frametitle{Grainsize For Extreme Scaling}
\begin{itemize}
 \item Strong Scaling is limited by expressed parallelism
 \begin{itemize}
  \item Minimum iteration time limited by lengthiest computation
  \begin {itemize} 
    \item Largest grains set lower bound
  \end{itemize}
 \end{itemize}
 \item 1-away generalized to k-away provides fine granularity control
\end{itemize}
\begin{centering}
\includegraphics[width=1.0\textwidth]{figures/1away2away}
\end{centering}
\end{frame}
%\end for class 

\begin{frame}[fragile]
\frametitle{NAMD: 2-AwayX Example}
\begin{centering}
\includegraphics[width=1.0\textwidth]{figures/2awayDiagramPlusHistos}
\end{centering}
\end{frame}

\begin{frame}
  \frametitle{Rules of thumb for grainsize}
  \begin{itemize}
    \item Make it as small as possible, as long as it amortizes the overhead
    \item More specifically, ensure:
      \begin{itemize}
      \item \textit{Average} grainsize is greater than $kv$ (say $10v$)
      \item No single grain should be allowed to be too large 
        \begin{itemize}
          \item Must be smaller than $\frac{T}{p}$, but actually we can express
            it as:
          \item Must be smaller than $kmv$ (say $100v$)
        \end{itemize}
      \end{itemize}
    \item Important corollary:
      \begin{itemize}
      \item You can be at close to optimal grainsize without having to think
        about $p$, the number of processors
      \end{itemize}
    \item $kv < g < mkv$ ($10v < g < 100v$)
  \end{itemize}
\end{frame}

% \begin{frame}
%   \frametitle{How to determine/ensure grainsize}
%   \begin{itemize}
%     \item Compiler techniques can help, but only in some cases
%       \begin{itemize}
%         \item Note that they don't need precise determination of grainsize,
%           just one that will satisfy a broad inequality
%       \end{itemize}
%   \end{itemize}
% \end{frame}

% \begin{frame}[fragile]
%   \frametitle{Performance of Fibonacci Example}
%   \begin{itemize}
%   \item How much work/computation does each object do in this example?
%   \item What are some of the overheads of this approach?
%   \item Is there way we can reduce/amortize the overhead?
%   \end{itemize}
% \end{frame}

\begin{frame}[fragile]
  \frametitle{Grain size for Fibonacci Example}
  \begin{itemize}
  \item Set a sequential threshold in the computational tree
    \begin{itemize}
    \item Past this threshold (i.e. when $n < threshold$), instead of
      constructing two new chares, compute the fibonacci sequentially
    \end{itemize}
  \end{itemize}
  \begin{center} \includegraphics[width=0.55\textwidth]{figures/tree-threshold.pdf} \end{center}
  \begin{itemize}
    \item Setting the grainsize limit at 4 (which is too small, but
      good for illustration)
    \item The internal nodes of the tree do very little work, but
    \item The coarser grains now amortize the cost of the fine-grained chares
  \end{itemize}
\end{frame}

% \begin{frame}[fragile]
%   \frametitle{Fibonacci w/Threshold Example}
%   \lstinputlisting[basicstyle=\tiny]{code/fib2.C}
% \end{frame}

\section[Charm++]{Structured Dagger}
%%%%%%%%%%%%%%%%%%%%%%%%%%%%%%%%%%%%
%%                                %%
%%       Structured Dagger        %%
%%                                %%
%%%%%%%%%%%%%%%%%%%%%%%%%%%%%%%%%%%%

\begin{frame}[fragile]
  \frametitle{Chares are reactive}
  \begin{itemize}
    \item The way we described Charm++ so far, a chare is a reactive entity:
      \begin{itemize}
      \item If it gets this method invocation, it does this action,
      \item If it gets that method invocation then it does that action
      \item But what does it do?
      \item In typical programs, chares have a \emph{life-cycle}
      \end{itemize}
    \item How to express the life-cycle of a chare in code?
      \begin{itemize}
      \item Only when it exists
        \begin{itemize}
        \item i.e. some chars may be truly reactive, and the programmer does
          not know the life cycle
        \end{itemize}
      \item But when it exists, its form is:
        \begin{itemize}
        \item Computations depend on remote method invocations, and completion
          of other local computations
        \item A DAG (Directed Acyclic Graph)!
        \end{itemize}
      \end{itemize}
  \end{itemize}
\end{frame}

\begin{frame}[fragile]
  \frametitle{Fibonacci Example}
  \lstinputlisting{code/fib1.ci}
\end{frame}

\begin{frame}[fragile]
  \frametitle{Fibonacci Example}
  \lstinputlisting[basicstyle=\tiny]{code/fib1.C}
\end{frame}


\begin{frame}
  \frametitle{Consider Fibonacci Chare}
  \begin{itemize}
  \item The Fibonacci chare gets created
  \item If its not a leaf,
    \begin{itemize}
    \item It fires two chares
    \item When both children return results (by calling \code{respond}):
      \begin{itemize}
      \item It can compute my result and send it up, or print it
      \end{itemize}
    \item But in our, this logic is hidden in the flags and counters $\ldots$
      \begin{itemize}
      \item This is simple for this simple example, but $\ldots$
      \end{itemize}
    \item Lets look at how this would look with a little notational support
    \end{itemize}
  \end{itemize}
\end{frame}

% \begin{frame}[fragile]
%   \frametitle{What is Structured Dagger?}
%   \begin{itemize}
%     \item Describe in a sequence the flow of control for a parallel object
%     \item Explicitly order and count message delivery and the blocks of code to
%       executed under the proper conditions
%   \end{itemize}
% \end{frame} 

%if -> while -> for
\begin{frame}[fragile]
  \frametitle{Structured Dagger}
  \framesubtitle{The \code{when} construct}
  \begin{itemize}
    \item The \code{when} construct
      \begin{itemize}
        \item Declare the actions to perform when a message is received
        \item In sequence, it acts like a blocking receive
      \end{itemize}
      \begin{lstlisting}[basicstyle=\normalsize]
entry void someMethod() {
  when entryMethod1(parameters) { /* block2 */ }
  when entryMethod2(parameters) { /* block3 */ }
};
      \end{lstlisting}
    \end{itemize}
\end{frame}

\begin{frame}[fragile]
  \frametitle{Structured Dagger}
  \framesubtitle{The \code{serial} construct}
  \begin{itemize}
    \item The \code{serial} construct
      \begin{itemize}
        \item A sequential block of C++ code in the .ci file
        \item The keyword \code{serial} means that the code block will be
          executed without interruption/preemption, like an entry method
        \item Syntax: \code{serial <optionalString> \{ /* C++ code */ \}}
        \item The \code{<optionalString>} is used for identifying the
          \code{serial} for performance analysis
        \item Serial blocks can access all members of the class they belong to
      \end{itemize}
    \item Examples (.ci file):
  \begin{columns}
    \begin{column}{0.5\textwidth}
      \begin{lstlisting}[basicstyle=\tiny]
entry void method1(parameters) {
  serial {
    thisProxy.invokeMethod(10);
    callSomeFunction();
  }
};
      \end{lstlisting}
    \end{column}
    \begin{column}{0.5\textwidth}
      \begin{lstlisting}[basicstyle=\tiny]
entry void method2(parameters) {
  serial "setValue" {
    value = 10;
  }
};
      \end{lstlisting}
    \end{column}
  \end{columns}
  \end{itemize}
\end{frame}


\begin{frame}[fragile]
  \frametitle{Structured Dagger}
  \framesubtitle{The \code{when} construct}
      \begin{lstlisting}[basicstyle=\tiny]
entry void someMethod() {
  serial { /* block1 */ }
  when entryMethod1(parameters) serial { /* block2 */ }
  when entryMethod2(parameters) serial { /* block3 */ }
};
      \end{lstlisting}
  \begin{itemize}
    \item Sequence
      \pause
      \begin{itemize}
        \item Sequentially execute \code{/* block1 */}
          \pause
        \item Wait for \code{entryMethod1} to arrive, if it has not, return control
          back to the Charm++ scheduler, otherwise, execute \code{/* block2 */}
          \pause
        \item Wait for \code{entryMethod2} to arrive, if it has not, return control
          back to the Charm++ scheduler, otherwise, execute \code{/* block3 */}
      \end{itemize}
    \end{itemize}
\end{frame}

\begin{frame}[fragile]
  \frametitle{Structured Dagger}
  \framesubtitle{The \code{when} construct}
  \begin{itemize}
  \item Execute \code{/* further sdag */} when \code{myMethod} arrives
  \begin{lstlisting}[basicstyle=\scriptsize]
when myMethod(int param1, int param2)
  /* further code */
  \end{lstlisting}

  \item Execute \code{/* further sdag */} when \code{myMethod1} and \code{myMethod2} arrive
  \begin{lstlisting}[basicstyle=\scriptsize]
when myMethod1(int param1, int param2),
      myMethod2(bool param3)
  /* further code */
  \end{lstlisting}

\item Which is almost the same as this:
  \begin{lstlisting}[basicstyle=\scriptsize]
when myMethod1(int param1, int param2) {
  when myMethod2(bool param3) { }
}
    /* further code */
  \end{lstlisting}

  \end{itemize}
\end{frame}

\begin{frame}
  \frametitle{Structured Dagger}
  \framesubtitle{Boilerplate}
  \begin{itemize}
    \item Structured Dagger can be used in any entry method (except for a constructor)
      \begin{itemize}
      \item Can be used in a \code{mainchare}, \code{chare}, or \code{array}
      \end{itemize}
    \item For any class that has Structured Dagger in it you must insert two
      calls:
       \begin{itemize}
         \item The Structured Dagger macro: \code{[ClassName]\_SDAG\_CODE}
%         \item Call the \code{\_\_sdag\_init()} initializer in the constructor
         \item For later: call the \code{\_\_sdag\_pup()} in the \code{pup} method
       \end{itemize}
  \end{itemize}
\end{frame}

\begin{frame}[fragile]
  \frametitle{Structured Dagger}
  \framesubtitle{Boilerplate}
  The .ci file:
  \begin{lstlisting}
    [mainchare,chare,array] MyFoo {
      ...
      entry void method(parameters) {
        // ... structured dagger code here ...
      };
      ...
    }
  \end{lstlisting}

  The .cpp file:
  \begin{lstlisting}
    class MyFoo : public CBase_MyFoo {
      MyFoo_SDAG_CODE /* insert SDAG macro */
    public:
      MyFoo() { }
    };
  \end{lstlisting}
\end{frame}

\begin{frame}[fragile]
  \frametitle{Fibonacci with Structured Dagger}
  \lstinputlisting[basicstyle=\scriptsize]{code/fibSDAG.ci}
\end{frame}

\begin{frame}[fragile]
  \frametitle{Fibonacci with Structured Dagger}
  \lstinputlisting[basicstyle=\tiny]{code/fibSDAG.cc}
\end{frame}

\begin{frame}[fragile]
  \frametitle{Structured Dagger}
  \framesubtitle{The \code{when} construct}
\begin{itemize}
  \item What is the sequence?
  \begin{lstlisting}[basicstyle=\scriptsize]
when myMethod1(int param1, int param2) {
  when myMethod2(bool param3),
        myMethod3(int size, int arr[size]) /* sdag block1 */
  when myMethod4(bool param4) /* sdag block2 */
}
  \end{lstlisting}
  \pause
  \item Sequence:
    \begin{itemize}
      \item Wait for \code{myMethod1}, upon arrival execute body of \code{myMethod1}
        \pause
      \item Wait for \code{myMethod2} and \code{myMethod3}, upon arrival of
        both, execute \code{/* sdag block1 */}
        \pause
      \item Wait for \code{myMethod4}, upon arrival execute \code{/* sdag block2 */}
    \end{itemize}
  \item Question: if \code{myMethod4} arrives first what will happen?
\end{itemize}

\end{frame}

% \begin{frame}[fragile]
%   \frametitle{Structured Dagger Constructs: Reference Numbers}
%   \begin{itemize}
%     \item Entry methods can be \emph{tagged} with a \emph{reference number}
%     \item A reference number is a special field in the envelope of the message
%       that is sent
%     \item By default, the reference number is a \code{short}
%     \item This can be changed when compiling charm:
%       \begin{itemize}
%       \item Add this to the build flags:
%         \code{--with-refnum-type=int}
%       \item For example, compiling on BG/P with the IBM XLC compiler:

%       \end{itemize}
%   \end{itemize}
% \code{./build charm++ bluegenep xlc --with-refnum-type=int -g -O0}

% \end{frame}

\begin{frame}[fragile]
  \frametitle{Structured Dagger Constructs}
  \framesubtitle{The \code{when} construct}
  \begin{itemize}
    \item The \code{when} clause can wait on a certain reference number
    \item If a reference number is specified for a \code{when}, the first
      parameter for the \code{when} must be the reference number
    \item Semantic: the \code{when} will ``block'' until a message arrives with
      that reference number
  \end{itemize}
  \begin{lstlisting}
    when method1[100](int ref, bool param1)
      /* sdag block */

    serial {
      proxy.method1(200, false); /* will not be delivered to the when */
      proxy.method1(100, true);  /* will be delivered to the when */
    }
  \end{lstlisting}
\end{frame}

\begin{frame}[fragile]
  \frametitle{Structured Dagger}
  \framesubtitle{The \code{if-then-else} construct}
  \begin{itemize}
  \item The \code{if-then-else} construct:
    \begin{itemize}
    \item Same as the typical C if-then-else semantics and syntax
    \end{itemize}
  \end{itemize}
  \begin{lstlisting}
  if (thisIndex.x == 10) {
    when method1[block](int ref, bool someVal) /* code block1 */
  } else {
    when method2(int payload) serial {
      //... some C++ code
    }
  }
  \end{lstlisting}
\end{frame}

\begin{frame}[fragile]
  \frametitle{Structured Dagger}
  \framesubtitle{The \code{for} construct}
  \begin{itemize}
  \item The \code{for} construct:
    \begin{itemize}
    \item Defines a sequenced \code{for} loop (like a sequential C for loop)
    \item Once the body for the $i$th iteration completes, the $i+1$ iteration
      is started
    \end{itemize}
  \end{itemize}
  \begin{lstlisting}
    for (iter = 0; iter < maxIter; ++iter) {
      when recvLeft[iter](int num, int len, double data[len])
        serial { computeKernel(LEFT, data); }
      when recvRight[iter](int num, int len, double data[len])
        serial { computeKernel(RIGHT, data); }
    }
  \end{lstlisting}
  \begin{itemize}
  \item \code{iter} must be defined in the class as a member
  \end{itemize}
  \begin{lstlisting}
    class Foo : public CBase_Foo {
      public: int iter;
    };
  \end{lstlisting}
\end{frame}

\begin{frame}[fragile]
  \frametitle{Structured Dagger}
  \framesubtitle{The \code{while} construct}
  \begin{itemize}
  \item The \code{while} construct:
    \begin{itemize}
    \item Defines a sequenced \code{while} loop (like a sequential C while loop)
    \end{itemize}
  \end{itemize}
  \begin{lstlisting}
    while (i < numNeighbors) {
      when recvData(int len, double data[len]) {
        serial {
          /* do something */
        }
        when method1() /* block1 */
        when method2() /* block2 */
      }
      serial { i++; }
    }
  \end{lstlisting}
\end{frame}

% \begin{frame}[fragile]
%   \frametitle{Structured Dagger Constructs: \code{when}}
%   \begin{itemize}
%   \item Another example:
%   \end{itemize}
%   .ci file:
%   \begin{lstlisting}
%     chare MyChare {
%       entry MyChare();
%       entry void startWork() {
%         serial { myRef = 100; }
%         when method1[myRef1](int ref, bool param1) /*  block1 */
%         when method2[myRef2](int ref, bool param1) /* block2 */
%       };
%     }
%   \end{lstlisting}
%   .cpp file:
%   \begin{lstlisting}
%     class MyChare : public CBase_MyChare {
%        int myRef1, myRef2;
%        MyChare() : myRef2(200) { }
%     };
%   \end{lstlisting}
% \end{frame}

\begin{frame}[fragile]
  \frametitle{Structured Dagger}
  \framesubtitle{The \code{overlap} construct}
  \begin{itemize}
  \item The \code{overlap} construct:
    \begin{itemize}
    \item By default, Structured Dagger defines a sequence that is followed sequentially
    \item \code{overlap} allows multiple independent clauses to execute in any order
    \item Any constructs in the body of an \code{overlap} can happen in any
      order
    \item An \code{overlap} finishes in sequence when all the statements in it
      are executed
    \item Syntax: \code{overlap \{ /* sdag constructs */ \}}
    \end{itemize}
  \end{itemize}
  What are the possible execution sequences?
  \begin{lstlisting}
    serial { /* block1 */ }
    overlap {
      serial { /* block2 */ }
      when entryMethod1[100](int ref_num, bool param1) /* block3 */
      when entryMethod2(char myChar) /* block4 */
    }
    serial { /* block5 */ }
  \end{lstlisting}
\end{frame}

\begin{frame}
  \frametitle{Illustration of a long ``overlap''}
  \begin{columns}
    \begin{column}{0.6\textwidth}
      \begin{itemize}
      \item Overlap can be used to get back some of the asynchrony within a chare
        \begin{itemize}
        \item But it is constrained
        \item Makes for more disciplined programming, 
          \begin{itemize}
          \item with fewer race conditions
          \end{itemize}
        \end{itemize}
      \end{itemize}
    \end{column}
    \begin{column}{0.4\textwidth}
      \includegraphics[width=0.8\textwidth]{figures/overlapFlow.png}
    \end{column}
  \end{columns}
\end{frame}

\begin{frame}[fragile]
  \frametitle{Structured Dagger}
  \framesubtitle{The \code{forall} construct}
  \begin{itemize}
  \item The \code{forall} construct:
    \begin{itemize}
    \item Has ``do-all'' semantics: iterations may execute an any order
    \item Syntax: \code{forall [<ident>] (<min> : <max>, <stride>) <body>}
    \item The range from \code{<min>} to \code{<max>} is inclusive
    \end{itemize}
  \end{itemize}
  \begin{lstlisting}
    forall [block] (0 : numBlocks - 1, 1) {
      when method1[block](int ref, bool someVal) /* code block1 */
    }
  \end{lstlisting}
  \begin{itemize}
    \item Assume \code{block} is declared in the class as \code{public: int block;}
  \end{itemize}
\end{frame}

\removeForTutorial{
\begin{frame}[allowframebreaks]
  \frametitle{Parallel Prefix with SDAG: .ci file}
  \lstinputlisting[basicstyle=\scriptsize]{code/par-prefix-sdag/prefix.ci}
\end{frame}

\begin{frame}[allowframebreaks]
  \frametitle{Parallel Prefix with SDAG: .C file}
  \lstinputlisting[basicstyle=\scriptsize]{code/par-prefix-sdag/prefix.C}
\end{frame}

}

% \begin{frame}[fragile]
%   \frametitle{Parallel Prefix with SDAG: .ci file}
%   \lstinputlisting{code/par-prefix-sdag/prefix.ci}
% \end{frame}
% 
% \begin{frame}[allowframebreaks]
%   \frametitle{Parallel Prefix with SDAG: .C file}
%   \lstinputlisting[basicstyle=\tiny]{code/par-prefix-sdag/prefix.C}
% \end{frame}

% \begin{frame}[fragile]
%   \frametitle{Determinant MP0 Solution: .ci file}
%   \lstinputlisting[basicstyle=\footnotesize]{code/det.ci}
% \end{frame}

% \begin{frame}[fragile]
%   \frametitle{Determinant MP0 Solution: .cpp file (part 1)}
%   \lstinputlisting[basicstyle=\tiny]{code/detp1.C}
% \end{frame}

% \begin{frame}[fragile]
%   \frametitle{Determinant MP0 Solution: .cpp file (part 2)}
%   \lstinputlisting[basicstyle=\tiny]{code/detp2.C}
% \end{frame}

% \begin{frame}[fragile]
%   \frametitle{Determinant MP0 Structered Dagger: .ci file}
%   \lstinputlisting[basicstyle=\scriptsize]{code/detSDAG.ci}
% \end{frame}

% \begin{frame}[fragile]
%   \frametitle{Determinant MP0 Structered Dagger: .cpp file}
%   \lstinputlisting[basicstyle=\tiny]{code/detSDAG.C}
% \end{frame}


\begin{frame}[fragile]
  \frametitle{Stencil Codes}
  \begin{center}
  \begin{itemize}
    \item Iterative applications where array elements are updated according to some fixed
    pattern.
    \item Used in computational simulations, solving partial differential equations, Jacobi kernel, Gauss–Seidel method, image processing applications etc.
    \item Can be 2D or 3D
  \end{itemize}
  \end{center}
\end{frame}

\begin{frame}[fragile]
  \frametitle{5-point Stencil}
   \begin{center} \includegraphics[width=0.85\textwidth]{figures/2DJacobi_Decomposition.jpg} \end{center}
\end{frame}

\begin{frame}[fragile]
  \frametitle{5-point Stencil}
   \begin{center} \includegraphics[width=0.6\textwidth]{figures/2DJacobi_NeighborComm.jpg} \end{center}
\end{frame}

\begin{frame}[fragile]
  \frametitle{5-point Stencil}
   \begin{center} \includegraphics[width=0.8\textwidth]{figures/2DJacobi_LogicFlow.jpg} \end{center}
\end{frame}

\begin{frame}[fragile]
  \frametitle{Jacobi: .ci file}
  \lstinputlisting[basicstyle=\footnotesize]{code/jacobi3dELL.ci}
\end{frame}

\begin{frame}[fragile]
  \frametitle{Jacobi: .ci file}
  \lstinputlisting[basicstyle=\tiny,linerange={22-49}]{code/jacobi3dSYNC.ci}
\end{frame}

\begin{frame}[fragile]
  \frametitle{Jacobi: .ci file (with \textbf{asynchronous} reductions)}
  \lstinputlisting[basicstyle=\tiny,linerange={22-50}]{code/jacobi3d.ci}
\end{frame}

% \begin{frame}[fragile]
%   \frametitle{Jacobi: .cpp file}
%   \lstinputlisting[basicstyle=\tiny,linerange={40-46,89-103}]{code/jacobi3d.C}
% \end{frame}

% \begin{frame}[fragile]
%   \frametitle{Jacobi: .cpp file}
%   \lstinputlisting[basicstyle=\tiny,linerange={109-139}]{code/jacobi3d.C}
% \end{frame}

% \begin{frame}[fragile]
%   \frametitle{Jacobi: .cpp file}
%   \lstinputlisting[basicstyle=\tiny,linerange={109-138}]{code/jacobi3d.C}
% \end{frame}

% \begin{frame}[fragile]
%   \frametitle{Jacobi: .cpp file}
%   \lstinputlisting[basicstyle=\tiny,linerange={164-190}]{code/jacobi3d.C}
% \end{frame}

% \begin{frame}[fragile]
%   \frametitle{Jacobi: .cpp file}
%   \lstinputlisting[basicstyle=\tiny,linerange={201-230}]{code/jacobi3d.C}
% \end{frame}

\begin{frame}[fragile]

  \frametitle{Power of Asynchrony}
  \framesubtitle{Example}

  \begin{itemize}
    \item Consider the following problem:
    \begin{itemize}
      \item A large number of key-value pairs are distributed on several (hundred) processors  (or chares)
      \pause
      \item Each chare needs to get some subset of these values before they can proceed to the next phase of the computation
      \pause 
      \item The set of keys needed are not known in advance: they are determined based on the input data
    \end{itemize}
  \end{itemize}
\end{frame}

\begin{frame}[fragile]
  \frametitle{Structured dagger version}
  \begin{lstlisting}[basicstyle=\footnotesize]
  for i=0,n {
      keys[i] = compute i’th key;
      KeyValues[keys[i]/B].requestValue( keys[i], thisProxy, i); 
  }
  \end{lstlisting}
  \pause
  \begin{lstlisting}[basicstyle=\footnotesize]
  for i= 0,n 
      when response(i,value)
          serial { values[i] = value;}

  // next phase of computation that’s uses the keys and values.
  \end{lstlisting}
  \pause
  \begin{lstlisting}[basicstyle=\footnotesize]
  KeyValueClass::requestValue(int key, Cproxy_Client c, int ref )
  {  
      ValueType v = localTable[key];
      c.response(ref, v); 
  }
  \end{lstlisting}
\end{frame}

% \begin{frame}[fragile]

%   \frametitle{Threaded EP version}

%   \begin{lstlisting}[basicstyle=\footnotesize]
%   for i=0,n {
%       keys[i] = compute i’th key;
%       futures[i] = CkCreateFuture();
%       KeyValues[keys[i]/B].requestViaFuture(futures[i], keys[i]); 
%   }
%   \end{lstlisting}
%   \pause
%   \begin{lstlisting}[basicstyle=\footnotesize]
%   for i= 0,n {
%       ValueMsg * m = CkWaitFuture(futures[i]);
%       values[i] = m->value;
%   }
%   // next phase of computation that’s uses the keys and values.
%   \end{lstlisting}
%   \pause
%   \begin{lstlisting}[basicstyle=\footnotesize]
%   KeyValueClass::requestViaFuture(CkFuture f , int key)
%   {  
%       ValueType v = localTable[key];
%       ValueMsg * m = new ValueMsg;
%       m->value = v;
%       CkSendToFuture(f,m);
%   }
%   \end{lstlisting}
% \end{frame}

% \begin{frame}[fragile]

%   \frametitle{MPI version}
%   \framesubtitle{Can be hard and inefficient}

%   \begin{lstlisting}[basicstyle=\footnotesize]
%   for i=0,n {
%       keys[i] = compute i’th key; 
%       MPI send request to processor keys[i]/B
%   }
%   for i= 0,n {
%       recv msg  containing the reference and value and store it
%   }
%   // next phase of computation that’s uses the keys and values.
%   \end{lstlisting}
%   \pause
%   \textcolor{red}{But this does not work because you have to process 
%   requests as well..And you don’t know how many requests there are.}
%   \pause
%   \\
%   \vspace{.2cm}
%   Other alternatives are possible, e.g. using 2 Alltoalls; but can be inefficient.
% \end{frame}



\section[Application Design]{Application Design}
\begin{frame}[t]
\frametitle{Molecular Dynamics in NAMD}
  \begin{columns}
  \column{.8\textwidth}
  \begin{itemize}
    \item Collection of charged atoms, with bonds
    \begin{itemize}
      \item Newtonian mechanics
      \item Relatively small of atoms (100K – 10M)
    \end{itemize}
    \pause
    \item Calculate forces on each atom 
    \begin{itemize}
      \item Bonds
      \item Non-bonded: electrostatic and van der Waal’s
      \begin{itemize}
        \item Short-distance: every timestep
        \item Long-distance: using PME (3D FFT)
        \item Multiple Time Stepping : PME every 4 timesteps 
      \end{itemize}
    \end{itemize}
    \pause
    \item Calculate velocities and advance positions
    \item Challenge: femtosecond time-step, millions needed!
  \end{itemize}
  \pause
  \textcolor{red}{Collaboration with K. Schulten, R. Skeel, and coworkers}
  \column{.2\textwidth}
  \vfill
  \begin{center} \includegraphics[width=\textwidth]{figures/namd.pdf} \end{center}
  \vfill
  \end{columns}
\end{frame}

\begin{frame}[t]
\frametitle{Spatial Decomposition Via Charm}
  \begin{columns}
  \column{.4\textwidth}
  \begin{center} \includegraphics[width=0.8\textwidth]{figures/namd_decomp.pdf} \end{center}
  \column{.6\textwidth}
  \begin{itemize}
    \item Atoms distributed to cubes based on their location
    \pause
    \item Size of each cube :
    \begin{itemize}
      \item Just a bit larger than cut-off radius
      \item Communicate only with neighbors
      \item Work: for each pair of nbr objects
    \end{itemize}
    \item C/C ratio: O(1)
    \pause
    \item However: 
    \begin{itemize}
      \item Load imbalance
      \item Limited parallelism
    \end{itemize}
  \end{itemize}
  \pause
  \textcolor{red}{Charm++ is useful to handle this case}
  \end{columns}
\end{frame}

\begin{frame}[t]
\frametitle{Object Based Parallelization for MD}
\framesubtitle{Force Decomposition + Spatial Decomposition}
  \begin{columns}
  \column{.4\textwidth}
  \begin{center} \includegraphics[width=0.8\textwidth]{figures/namd_decomp2.pdf} \end{center}
  \column{.6\textwidth}
  \begin{itemize}
    \item Now, we have many objects to load balance:
    \begin{itemize}
      \item Each diamond can be  assigned to any proc.
      \item Number of diamonds (3D): 14*Number of Patches
    \end{itemize}
    \pause
    \item 2-away variation:
    \begin{itemize}
      \item Half-size cubes
      \item Communicate only with neighbors
      \item 5 x 5 x 5 interactions
    \end{itemize}
    \pause
    \item 3-away interactions: 7 x 7 x 7
  \end{itemize}
  \end{columns}
\end{frame}

\begin{frame}[t]
\frametitle{NAMD Parallelization Using Charm++}
The computation is decomposed into ``natural'' objects of the application, which
are assigned to processors by Charm++ RTS
  \begin{center} \includegraphics[width=\textwidth]{figures/md_parallelize.pdf} \end{center}
\end{frame}

\begin{frame}[t]
\frametitle{NAMD Projections}
  \begin{center} \includegraphics[width=.9\textwidth]{figures/namd_projection.pdf} \end{center}
\end{frame}

\begin{frame}[t]
\frametitle{DHFR Performance on Titan}
  \begin{itemize}
  \item Best performance is 590us/step
  \end{itemize}
  \begin{center} \includegraphics[width=.8\textwidth]{figures/jac-titan-pme4.pdf} \end{center}
\end{frame}

\begin{frame}[t]
\frametitle{Apoa1 Performance on BG/P BG/Q}
  \begin{itemize}
  \item Best performance on BG/Q is 794us/step
  \end{itemize}
  \begin{center} \includegraphics[width=.8\textwidth]{figures/apoa1-pme4-PQ.pdf} \end{center}
\end{frame}

\begin{frame}[t]
\frametitle{NAMD Performance on IBM Blue Gene/P}
  \begin{center} \includegraphics[width=.8\textwidth]{figures/namd_bgp.pdf} \end{center}
\end{frame}

\begin{frame}[t]
\frametitle{100M STMV Performance on Titan}
  \begin{center} \includegraphics[width=.8\textwidth]{figures/namd_titan.pdf} \end{center}
\end{frame}

\begin{frame}[t]
\frametitle{ChaNGa: Parallel Gravity}
  \begin{itemize}
    \item Collaborative project (NSF)
    \begin{itemize} 
        \item with Tom Quinn, Univ. of Washington
    \end{itemize}
    \pause
    \item Evolution of Universe and Galaxy Formation
    \item Gravity, gas dynamics
    \pause
    \item Barnes-Hut tree codes
    \begin{itemize} 
      \item Oct tree is natural decomposition
      \item Geometry has better aspect ratios, so you ``open” up fewer nodes
      \item But is not used because it leads to bad load balance
      \item Assumption: one-to-one map between sub-trees and PEs
      \item Binary trees are considered better load balanced
    \end{itemize}
    \pause
    \item With Charm++: Use Oct-Tree, and let Charm++ map subtrees to processors
  \end{itemize}
\end{frame}

\begin{frame}[t]
\frametitle{ChaNGa: Control Flow}
  \begin{center} \includegraphics[width=\textwidth]{figures/changa.pdf} \end{center}
\end{frame}

\begin{frame}[t]
\frametitle{OpenAtom: MD with quantum effects}
  \begin{columns}
  \column{.5\textwidth}
    \begin{itemize}
      \item Much more fine-grained:
      \begin{itemize}
        \item Each electronic state is modeled with a large array
      \end{itemize}
      \pause
      \item Collaboration with:
      \begin{itemize}
        \item G. Martyna (IBM) 
        \item M. Tuckerman (NYU)
      \end{itemize}
      \pause
    \item Using Charm++ virtualization, we can efficiently scale small (32 molecule) systems to thousands of processors
  \end{itemize}
  \column{.5\textwidth}
  \begin{center} \includegraphics[width=.5\textwidth]{figures/openatom1.png}\\
  \textcolor{red}{Semiconductor Surfaces}\end{center}
  \begin{center} \includegraphics[width=.5\textwidth]{figures/openatom2.png}\\
  \textcolor{red}{Nanowires}\end{center}
  \end{columns}
\end{frame}


\begin{frame}[t]
\frametitle{OpenAtom: Decomposition and Computation Flow}
  \begin{center} \includegraphics[width=\textwidth]{figures/openatom_array.pdf} \end{center}
\end{frame}

% \begin{frame}[t]
% \frametitle{OpenAtom: Topology Aware Mapping of Objects}
%   \begin{center} \includegraphics[width=.45\textwidth]{figures/openatom_topo.pdf} \end{center}
%   Object based decomposition provides new degrees of freeedom to easily try
%   different mappings of objects to processors, to help minimize contention

% \end{frame}

\section[Tuning]{Performance Tuning}
\begin{frame}
  \frametitle{Performance Analysis Using Projections}
  \begin{itemize}
  \item Instrumentation and measurement
  \begin{itemize}
  \item Link program with {\tt -tracemode projections or summary}
  \item Trace data is generated automatically during run
  \item User events can be easily inserted as needed
  \end{itemize}
  \item Projections: visualization and analysis
  \begin{itemize}
  \item Scalable tool to analyze up to 300,000 log files
  \item A rich set of tool features : time profile, time lines, usage profile, histogram, extrema tool
  \item Detect performance problems: load imbalance, grain size, communication bottleneck, etc 
  \end{itemize}
  \end{itemize}

\end{frame}

\begin{frame}
\frametitle{Using Projections }
 \begin{itemize}
  \item Tools of aggregated performance viewing
   \begin{itemize}
    \item Time profile
    \item Histogram
    \item Communication over time
   \end{itemize}
  \item Tools of processor level granularity
  \begin{itemize}
   \item Overview
   \item Timeline
  \end{itemize}
  \item Tools of derived/processed data
  \begin{itemize}
   \item Extrema analysis : identifies outliers
   \item Noise miner : highlights probable interference
  \end{itemize}
 \end{itemize}
\end{frame}

\begin{frame}
\frametitle{Problem Identification }
\begin{itemize}
 \item Load imbalance
 \begin{itemize}
  \item Time profile : lower CPU usage
  \item Extrema analysis tool:
  \begin{itemize}
   \item Least idle processors
  \end{itemize}
  \item Load the over-loaded processors in Timeline
  \item Histogram : grain size issues  
 \end{itemize}
\end{itemize}
\end{frame}

\begin{frame}
\frametitle{Using Projections }
\begin{itemize}
\item Example Demonstration
\begin{itemize}
\item Trying to identify the next performance obstacle for NAMD
\begin{itemize}
\item Running on 8192 processors, with 1 million atom simulation
\item Jaguar Cray XK6 
\item Test scenario: with PME every step
\end{itemize}
\end{itemize}
\end{itemize}
\end{frame}

\begin{frame}{Time Profile}
 \includegraphics<1>[width=0.9\textwidth]{figures/prj1M8KTimeprofile.png}
\end{frame}

\begin{frame}{Extrema Tool for Least Idle Processors}
\includegraphics<1>[width=0.9\textwidth]{figures/prj1M8KExtrema.png}
\end{frame}

\begin{frame}{Time Lines with Message Back Tracing}
\centering
\includegraphics<1>[width=0.9\textwidth]{figures/prj1M8KTimeline.png}
\end{frame}


\begin{frame}{Communication over Time for all Processors}
 \includegraphics<1>[width=0.9\textwidth]{figures/prj1M8KCommtime.png}
\end{frame}

\section[LB]{Using Dynamic Load Balancing}
\transition{The PUP Framework}
\begin{frame}[fragile]
\frametitle{How to Diagnose Load Imbalance}
\begin{itemize}
 \item Often hidden in statements such as:
 \begin{itemize}
  \item Very high synchronization overhead
  \begin{itemize}
   \item Most processors are waiting at a reduction
  \end{itemize}
 \end{itemize}
 \item Count total amount of computation (ops/flops) per processor
 \begin{itemize}
  \item In each phase! 
  \item Because the balance may change from phase to phase
 \end{itemize}
\end{itemize}
\end{frame}

\begin{frame}[fragile]
\frametitle{Golden Rule of Load Balancing}
\emph{Fallacy: objective of load balancing is to minimize variance in load across processors}

\begin{itemize}
 \item[]\emph{Example:}
 \begin{itemize}
  \item  50,000 tasks of equal size, 500  processors:
  \begin{itemize}
   \item A: All processors get 99, except last 5 gets $100+99 = 199$
   \item OR, B:  All processors have 101, except last 5 get 1
  \end{itemize}
 \end{itemize}
 \item[] Identical variance, but situation A is much worse!
\end{itemize}


\emph{Golden Rule: It is ok if a few processors idle, but avoid having processors that are overloaded with work}


\emph{Finish time} = maxi(Time on processor i)
\begin{itemize}
\item[] excepting data dependence and communication overhead issues
\end{itemize}

The speed of any group is the speed of slowest member of that group.
\end{frame}

\begin{frame}[fragile]
\frametitle{Automatic Dynamic Load Balancing}
\begin{itemize}
\item Measurement based load balancers
\begin{itemize}
\item Principle of persistence: In many CSE applications, computational loads and communication patterns tend to persist, even in dynamic computations
\item Therefore, recent past is a good predictor of near future
\item Charm++ provides a suite of load-balancers 
\item Periodic measurement and migration of objects
\end{itemize}
\item Seed balancers (for task-parallelism)
\begin{itemize}
\item Useful for divide-and-conquer and state-space-search applications
\item Seeds for charm++ objects moved around until they take root
\end{itemize}
\end{itemize}
\end{frame}

\begin{frame}[fragile]
\frametitle{Using the Load Balancer}
\begin{itemize}
\item link a LB module 
\begin{itemize}
\item \code{-module <strategy>}
\item RefineLB, NeighborLB, GreedyCommLB, others…
\item EveryLB will include all load balancing strategies
\end{itemize}
\item compile time option (specify default balancer)
\begin{itemize}
\item \code{-balancer RefineLB}
\item runtime option
\item \code {+balancer RefineLB}
\end{itemize}
\end{itemize}
\end{frame}

\begin{frame}
\frametitle{Code to Use Load Balancing}
\begin{itemize}
\item Insert \code{if (myLBStep) AtSync() else ResumeFromSync();} call at natural barrier
\item Implement \code{ResumeFromSync()} to resume execution
\begin{itemize}
\item Typical ResumeFromSync contribute to a reduction
\end{itemize}
\end{itemize}
\end{frame}

\begin{frame}[fragile]
\frametitle{Example: Stencil}
\begin{lstlisting}[basicstyle=\tiny]
while (!converged) {
  serial {
    int x = thisIndex.x, y = thisIndex.y, z = thisIndex.z;
    copyToBoundaries();
    thisProxy(wrapX(x-1),y,z).updateGhosts(i, RIGHT, dimY, dimZ, right);
    /* ...similar calls to send the 6 boundaries... */
    thisProxy(x,y,wrapZ(z+1)).updateGhosts(i, FRONT, dimX, dimY, front);
  }
  for (remoteCount = 0; remoteCount < 6; remoteCount++) {
    when updateGhosts[i](int i, int d, int w, int h, double b[w*h])
    serial { updateBoundary(d, w, h, b); }
  }
  serial {
    int c = computeKernel() < DELTA;
    CkCallback cb(CkReductionTarget(Jacobi, checkConverged), thisProxy);
    if (i%5 == 1) contribute(sizeof(int), \&c, CkReduction::logical_and, cb);
  }
  if (i % lbPeriod == 0) { serial { AtSync(); } when ResumeFromSync() {} }
  if (++i % 5 == 0) {
    when checkConverged(bool result) serial {
      if (result) { mainProxy.done(); converged = true; }
    }
  }
}
\end{lstlisting}
\end{frame}

\begin{frame}
\frametitle{Performance}
\begin{centering}
\includegraphics[width=0.5\textwidth]{figures/beforeLB}
\includegraphics[width=0.5\textwidth]{figures/afterLB}
\end{centering}
\end{frame}

\begin{frame}[fragile]
\frametitle{Grainsize and Load Balancing}
\begin{itemize}
\item[] How Much Balance Is Possible?
\end{itemize}
\begin{centering}
\includegraphics[width=0.8\textwidth]{figures/histogramGrains}
\end{centering}
\end{frame}

\begin{frame}[fragile]
\frametitle{Grainsize For Extreme Scaling}
\begin{itemize}
 \item Strong Scaling is limited by expressed parallelism
 \begin{itemize}
  \item Minimum iteration time limited lengthiest computation
  \begin {itemize} 
    \item Largest grains set lower bound
  \end{itemize}
 \end{itemize}
 \item 1-away generalized to k-away provides fine granularity control
\end{itemize}
\begin{centering}
\includegraphics[width=1.0\textwidth]{figures/1away2away}
\end{centering}
\end{frame}

\begin{frame}[fragile]
\frametitle{NAMD: 2-AwayX Example}
\begin{centering}
\includegraphics[width=1.0\textwidth]{figures/2awayDiagramPlusHistos}
\end{centering}
\end{frame}

\begin{frame}[fragile]
\frametitle{Load Balancing Strategies}
\begin{itemize}
 \item Classified by when it is done:
 \begin{itemize}
  \item Initially
  \item Dynamic: Periodically
  \item Dynamic: Continuously
 \end{itemize}
 \item Classified by whether decisions are taken with global information
 \begin{itemize}
  \item Fully centralized
  \begin{itemize}
   \item Quite good a choice when load balancing period is high
  \end{itemize}
 \item Fully distributed
 \begin{itemize}
  \item Each processor knows only about a constant number of neighbors
  \item Extreme case: totally local decision (send work to a random destination processor, with some probability).
 \end{itemize}
\item Use \emph{aggregated} global  information, and \emph{detailed} neighborhood info.
\end{itemize}
\end{itemize}
\end{frame}

\begin{frame}[fragile]
\frametitle{Dynamic Load Balancing Scenarios}
\begin{itemize}
 \item Examples representing typical classes of situations
 \begin{itemize}
  \item Particles distributed over simulation space
  \begin{itemize}
   \item Dynamic: because Particles move.
   \begin{description}
   \item Highly non-uniform distribution (cosmology)
   \item Relatively Uniform distribution 
   \end{description}
  \end{itemize}
 \end{itemize}
\item Structured grids, with dynamic refinements/coarsening
\item Unstructured grids with dynamic refinements/coarsening
\end{itemize}
\end{frame}

\begin{frame}[fragile]
\frametitle{Example Case: Particles}
 Orthogonal Recursive Bisection (ORB)
 %should wrapfig the little diagram here
 \begin{itemize} 
  \item At each stage: divide Particles equally
  \item Processor don’t need to be a power of 2:
  \begin{itemize} 
   \item Divide in proportion 
   \begin{itemize} 
    \item 2:3 with 5 processors
   \end{itemize}
  \end{itemize}
 \item How to choose the dimension along which to cut?
 \begin{itemize} 
  \item Choose the longest one
 \end{itemize}
 \item How to draw the line?
 \begin{itemize} 
  \item All data on one processor? Sort along each dimension
  \item Otherwise: run a distributed histogramming algorithm to find the line, recursively
 \end{itemize}
 \item Find the entire tree, and then do all data movement at once
 \begin{itemize} 
  \item Or do it in two-three steps.
  \item But no reason to redistribute particles after drawing each line.
 \end{itemize}
\end{itemize}
\end{frame}

\begin{frame}[fragile]
\frametitle{Dynamic Load Balancing using Objects}

Object based decomposition (I.e. virtualized decomposition) helps
\begin{itemize}
 \item Allows RTS to remap them to balance load
 \item But how does the RTS decide where to map objects?
 \item Just move objects away from overloaded processors to underloaded processors
 \item How is load determined?
\end{itemize}
\end{frame}

\begin{frame}[fragile]
\frametitle{Measurement Based Load Balancing}
\begin{itemize}
 \item \emph{Principle of Persistence}
 \begin{itemize}
  \item Object communication patterns and computational loads tend to persist over time
  \item In spite of dynamic behavior
  \begin{itemize}
   \item Abrupt but infrequent changes
   \item Slow and small changes
  \end{itemize}
 \end{itemize}
\item Runtime instrumentation
\begin{itemize}
 \item Measures communication volume and computation time
\end{itemize}
\item Measurement based load balancers
\begin{itemize}
 \item Use the instrumented data-base periodically to make new decisions
 \item Many alternative strategies can use the database
\end{itemize}
\end{itemize}
\end{frame}

\begin{frame}[fragile]
\frametitle{Periodic Load Balancing}

Stop the computation?

Centralized strategies:
\begin{itemize}
 \item Charm RTS collects data (on one processor) about:
 \begin{itemize}
  \item Computational Load and Communication for each pair
 \end{itemize}
 \item If you are not using AMPI/Charm, you can do the same instrumentation and data collection
 \item Partition the graph of objects across processors
 \begin{itemize}
  \item Take communication into account
  \begin{itemize}
   \item Pt-to-pt, as well as multicast over a subset
   \item As you map an object, add to the load on both sending and receiving processor
  \end{itemize}
  \item Multicasts to multiple co-located objects are effectively the cost of a single send
 \end{itemize}
\end{itemize}
\end{frame}

\begin{frame}[fragile]
\frametitle{Typical Load Balancing Steps}
\begin{centering}
\includegraphics[width=1.0\textwidth]{figures/LBStepsDiagram}
\end{centering}
\end{frame}

\begin{frame}[fragile]
\frametitle{Object Partitioning Strategies}
\begin{itemize}
 \item You can use graph partitioners like METIS, K-R
 \begin{itemize}
  \item BUT: graphs are smaller, and optimization criteria are different
 \end{itemize}
 \item Greedy strategies:
 \begin{itemize}
  \item If communication costs are low: use a simple greedy strategy
  \begin{itemize}
   \item Sort objects by decreasing load
   \item Maintain processors in a heap (by assigned load)
   \item In each step: 
   \begin{description}
    \item assign the heaviest remaining object to the least loaded processor
   \end{description}
  \end{itemize}
  \item With small-to-moderate communication cost:
  \begin{itemize}
   \item Same strategy, but add communication costs as you add an object to a processor
  \end{itemize}
  \item Always add a refinement step at the end:
  \begin{itemize}
   \item Swap work from heaviest loaded processor to ``some other processor''
   \item Repeat a few times or until no improvement 
  \end{itemize}
 \end{itemize}
\end{itemize}
\end{frame}


\begin{frame}[fragile]
\frametitle{Object Partitioning Strategies 2}
When communication cost is significant:
\begin{itemize}
 \item Still use greedy strategy, but:
 \begin{itemize}
  \item At each assignment step, choose between assigning O to least loaded processor and the processor that already has objects that communicate most with O.
  \begin{itemize}
   \item Based on the degree of difference in the two metrics
   \item Two-stage assignments:
   \begin{description}
    \item In early stages,  consider communication costs as long as the processors are in the same (broad) load “class”,
    \item In later stages, decide based on load
   \end{description}
  \end{itemize}
 \end{itemize}
\end{itemize}
Branch-and-bound
\begin{itemize}
 \item Searches for optimal, but can be stopped after a fixed time
\end{itemize}
\end{frame}

\begin{frame}[fragile]
\frametitle{Crack Propagation}
\begin{centering}
\includegraphics[width=0.4\textwidth]{figures/chunkGraph16}
\includegraphics[width=0.4\textwidth]{figures/chunkGraph128}
\end{centering}
Decomposition into 16 chunks (left) and 128 chunks, 8 for each PE (right). The middle area contains cohesive elements. Both decompositions obtained using Metis. Pictures: S. Breitenfeld, and P. Geubelle

As computation progresses, crack propagates, and new elements are added, leading to more complex computations in some chunks
\end{frame}

\begin{frame}[fragile]
\frametitle{Load Balancing Crack Propagation}
\begin{centering}
\includegraphics[width=1.0\textwidth]{figures/LButilizationCrackPropWithAnnotation}
\end{centering}
\end{frame}


\begin{frame}[fragile]
\frametitle{Distributed Load balancing}
\begin{itemize}
 \item Centralized strategies
 \begin{itemize}
  \item Still ok for 3000 processors for NAMD
 \end{itemize}
 \item Distributed balancing is needed when:
 \begin{itemize}
  \item Number of processors is large and/or 
  \item load variation is rapid
 \end{itemize}
 \item Large machines: 
 \begin{itemize}
  \item Need to handle locality of communication
  \begin{itemize}
   \item Topology sensitive placement
  \end{itemize}
  \item Need to work with scant global information
  \begin{itemize}
   \item Approximate or aggregated global information (average/max load)
   \item Incomplete global info (only “neighborhood”)
   \item Work diffusion strategies (1980s work by Kale and others!)
  \end{itemize}
  \item Achieving global effects by local action…
 \end{itemize}
\end{itemize}
\end{frame}

\begin{frame}[fragile]
\frametitle{Load Balancing on Large Machines}
\begin{itemize}
 \item Existing load balancing strategies don’t scale on extremely large machines
 \item Limitations of centralized strategies:
 \begin{itemize}
  \item Central node: memory/communication bottleneck
  \item Decision-making algorithms tend to be very slow
 \end{itemize}
 \item Limitations of distributed strategies:
 \begin{itemize}
  \item Difficult to achieve well-informed load balancing decisions
 \end{itemize}
\end{itemize}
\end{frame}


\begin{frame}[fragile]
\frametitle{Simulation Study - Memory Overhead}
lb\_test experiments performed with the performance simulator BigSim
\includegraphics[width=1.0\textwidth]{figures/LBMemOverhead}
\begin{itemize}
 \item lb\_test benchmark is a parameterized program that creates a 
specified number of communicating objects in 2D-mesh.
\end{itemize}
\end{frame}


\begin{frame}[fragile]
\frametitle{Hierarchical Load Balancers}
\begin{itemize}
\item Partition processor allocation into processor groups
\item Apply different strategies at each level
\item Scalable to a large number of processors
\end{itemize}
\end{frame}

\begin{frame}[fragile]
\frametitle{Our Hybrid Scheme}
\includegraphics[width=1.0\textwidth]{figures/hybridLBScheme}
\end{frame}

\begin{frame}[fragile]
\frametitle{Hybrid Load Balancing Performance}
\includegraphics[width=0.5\textwidth]{figures/hybridLBPerf}
\includegraphics[width=0.5\textwidth]{figures/hybridLBquality}
\end{frame}

\section[Fault Tol]{Checkpointing and Resilience}
\transition{Resilience}
\begin{frame}[fragile]

  \frametitle{Fault Tolerance in Charm++/AMPI}

  \begin{itemize}
    \item Four Approaches:

      \begin{itemize}
\item Disk-based checkpoint/restart
\item In-memory double checkpoint/restart
\item Experimental: Proactive object migration 
\item Experimental: Message-logging for scalable fault tolerance
      \end{itemize}
    \item Common Features:
\begin{itemize}
\item Easy checkpoint: 
\item migrate-to-disk leverages object-migration capabilities
\item Based on dynamic runtime capabilities
\item Can be used in concert with load-balancing schemes
     \end{itemize}
  \end{itemize}
\end{frame}

\begin{frame}[fragile]
  \frametitle{Checkpointing to the file system : Split Execution}
\begin{itemize}
\item The common form of checkpointing
\begin{itemize}
\item The job runs for 5 hours, then will continue at the next
  allocation another day! 
\end{itemize}

\item The existing Charm++ infrastructure for chare migration helps:
\item Just ``migrate'' chares to disk
\item The call to checkpoint the application is made at a quiescent
  point from one of the processor

\item CHECK IF THIS IS TRUE
\end{itemize}

 \begin{lstlisting}[basicstyle=\footnotesize]

. . .  CkCallback cb(CkIndex_Hello::SayHi(),helloProxy);
       CkStartCheckpoint(``log'',cb);

> ./charmrun hello +p4 +restart log

\end{lstlisting}

\end{frame}

\begin{frame}[fragile]
  \frametitle{In-mempory checkplinting with auto restart}
\begin{itemize}

\item Idea: checkpoint data in a buddy processor's memory, in addition
  to a local checkpoint
\item System auto detects when a node crashes
\item  Failed process is restarted on a spare, and retrieves it
  checkpoint form the buddy
\item (you can also do without the spare)
\item Every other processor retrieves its local checkpoint 
\end{itemize}

 \begin{lstlisting}[basicstyle=\footnotesize]

void CkStartMemCheckpoint(CkCallback &cb)
\end{lstlisting}

\end{frame}

\begin{frame}[fragile]
  \frametitle{Get plots : slides 51 and 52 from LBNL pptx }
\end{frame}


\section[AMPI & Interop]{Interoperability}
\begin{frame}[fragile]

  \frametitle{Adaptive MPI}

  \begin{itemize}
    \item MPI implemented on top of Charm++ 
    \item Each MPI process implemented as a user-level thread embedded in a
    chare
    \item Overdecompose to obtain communication-computation overlap between
    threads
    \item Supports migration, load balancing, fault tolerance and other Charm++
    functionality
    \item Use cases - Rocstar, BRAMS, NPB, Lulesh etc
    \item Build with AMPI as target and compile using ampi* compilers
  \end{itemize}
\end{frame}

\begin{frame}[fragile]

  \frametitle{Charm++-MPI Interoperability}

  \begin{itemize}
    \item Any library written in Charm++ can be called from MPI
    \pause
    \item Charm++ resides in the same memory space as the MPI program
    \item Control transfer between MPI and Charm++ analogous to the control transfer
    between a program and an external library being used by the program
    \pause
    \item Currently requires mpi-based build of Charm++

  \end{itemize}
\end{frame}

\begin{frame}[fragile]
  \frametitle{Interoperability Modes}
    \includegraphics[width=0.45\textwidth,height=8cm]{figures/newInterop_a}
    \pause
    \includegraphics[width=0.275\textwidth,height=8cm]{figures/newInterop_b}
    \pause
    \includegraphics[width=0.275\textwidth,height=8cm]{figures/newInterop_c}
\end{frame}

\begin{frame}[fragile]
  \frametitle{Example Code Flow}
  \begin{lstlisting}[basicstyle=\footnotesize]
  MPI_Init(argc,argv); //initialize MPI
  //Do MPI related work here
  \end{lstlisting}
  \pause
  \begin{lstlisting}[basicstyle=\footnotesize]
  //create comm to be used by Charm++
  MPI_Comm_split(MPI_COMM_WORLD, myRank % 2, myRank, newComm); 
  CharmLibInit(newComm,….) //initialize Charm++ over my communicator
  \end{lstlisting}
  \pause
  \begin{lstlisting}[basicstyle=\footnotesize]
  if(myRank % 2) 
    StartHello(…); //invoke Charm++ library on one set
  else 
    //do MPI work on other set
  \end{lstlisting}
  \pause
  \begin{lstlisting}[basicstyle=\footnotesize]
  kNeighbor(…);  //invoke Charm++ library on both sets individually
  CharmLibExit();  //destroy Charm++ 
  \end{lstlisting}
\end{frame}

\begin{frame}[fragile]

  \frametitle{Enabling Interoperability}

  \begin{itemize}
    \item Add interface functions that can be called from MPI, and triggers
    Charm++ RTS-
    \begin{lstlisting}[basicstyle=\footnotesize]
    void StartHello(int elems)
      if(CkMyPe() == 0) {
        CProxy_MainHello mainhello =
        CProxy_MainHello::ckNew(elems); 
      }
      StartCharmScheduler();
    }
    \end{lstlisting}
  \pause
  \item Use CkExit to return the control back to MPI
  \item Include {\em mpi-interoperate.h} in MPI and Charm++ code
  \end{itemize}
\end{frame}




\section[Debugging]{Debugging}
\begin{frame}[fragile]
  \frametitle{Overview}
  %\includegraphics[width=0.9\textwidth]{figure/overviewDebug.png}
\end{frame}

\input{heterogeneity}
\section[Messages, Groups, Shared Mem]{Further Optimization}
\transition{Overview of Performance Enhancement Features} 
%\transition{Advanced Material}
\begin{frame}[fragile]
  \frametitle{Shared Memory Optimizations}
  \begin{itemize}
    \item Objects' memory buffers disjoint
    \item Communication will leverage refcounted message pointers to avoid copying
    \item Avoids packing/unpacking within node
    \item Single copy of node level read only structures
    \item Dedicated thread for intra-node communication
  \end{itemize}
\end{frame}

\begin{frame}[fragile]
  \frametitle{Controlling Placement}
  \begin{itemize}
    \item In some applications, load patterns don’t change much as computation progresses
    \begin{itemize}
      \item You, the programmer, may want to control which chare lives on which processors
      \item This is also true when  load may evolve over time, but you want to control initial placement of chares
    \end{itemize}
    \item The feature in Charm++ for this purpose is called Map Objects
    \begin{itemize}
      \item Sec. 13.2.2 of the Charm++ manual
    \end{itemize}
  \end{itemize}
\end{frame}

\begin{frame}[fragile]
  \frametitle{Messages}
  \begin{itemize}
    \item Avoids extra copy
    \item Can be custom packed
    \item Reusable
    \item Useful for transfer of complex data structures
    \item It provides explicit control for the application over allocation, reuse, and scope
    \item Encapsulates variable size quantities
    \item Execution order of messages in the queue can be prioritized
  \end{itemize}
\end{frame}

\begin{frame}[fragile]
  \frametitle{Groups}
  \begin{itemize}
    \item Like a chare-array with one chare per PE
    \item Encapsulate processor local data
    \item May access the local member as a regular C++ object
    \item In .ci file, 
    \begin{lstlisting}
group ExampleGroup {
  // Interface specifications as for normal chares
  // For instance, the constructor ...
  entry ExampleGroup(parameters1);
  // ... and an entry method
  entry void someEntryMethod(parameters2);
};
    \end{lstlisting}
    \item No difference in .h and .C file definitions
  \end{itemize}
\end{frame}

\begin{frame}[fragile]
  \frametitle{Node Groups}
  \begin{itemize}
    \item A chare-array with one chare per node
    \begin{itemize}
      \item In non-smp node groups and node groups are same
    \end{itemize}
    \item No difference in .h and .C
    \item Creation and usage same as others
    \item An entry method on a node-group member may be executed on any PE of the node
    \item Concurrent execution of two entry methods of a node-group member may happen
    \begin{itemize}
      \item Use \code{[exclusive]} for entry methods which are unsuitable for reentrance safety
    \end{itemize}
  \end{itemize}
\end{frame}

\begin{frame}[fragile]
  \frametitle{Customizing Entry Method Attributes}
  \begin{itemize}
    \item \code{threaded} – executed using separate thread
    \begin{itemize}
      \item each thread has a stack, and may be suspended, for sync methods or futures
      \item to set stack’s size use +stacksize $<$ size in bytes $>$
    \end{itemize}
    \item \code{sync} - returns a value
    \item \code{inline} – entry method invoked immediately if destination chare on same PE
    \begin{itemize}
      \item blocking call
    \end{itemize}
    \item \code{reductiontarget} – target of an array reduction
    \begin{itemize}
      \item Takes parameter marshaled arguments
    \end{itemize}
    \item \code{notrace} – not traced for projections
  \end{itemize}
\end{frame}

\begin{frame}[fragile]
  \frametitle{Customizing Entry Methods}
  \begin{itemize}
    \item \code{expedited} – entry method skips the priority-based message queue in Charm++ runtime (for groups)
    \item \code{immediate} - skips the message scheduling queue (for any chare array)
    \item \code{nokeep} – message belongs to Charm
    \item \code{exclusive} – mutual exclusion on execution of entry methods on node-groups 
    \item \code{python} – can be called from python scripts
  \end{itemize}
\end{frame}

\begin{frame}[fragile]
\frametitle{Sections}
\begin{itemize}
 \item It is often convenient to define subcollections of elements within a
   chare array
   \begin{itemize}
   \item Example: rows or columns of a 2D chare array
   \item One may wish to perform collective operations on the subcollection
     (e.g. broadcast, reduction)
   \end{itemize}
 \item Sections are the standard subcollection construct in Charm++

\end{itemize}
\begin{lstlisting}
CProxySection_Hello proxy =
  CProxySection_Hello::ckNew(helloArrayID, 0, 9, 1, 0, 19, 2, 0, 29, 2);
\end{lstlisting}
\end{frame}

\begin{frame}[fragile]
\frametitle{\code{sync} methods}
\begin{itemize}
 \item Synchronous as opposed to asynchronous
 \item They return a value - always a \code{message} type
 \item Other than that, just like any other entry method:
\end{itemize}
In the interface file:
\begin{lstlisting}
  entry [sync] MsgData * f(double A[2*m], int m ); 
\end{lstlisting}

In the C++ file:
\begin{lstlisting}
MsgData *f(double X[], int size) {
  ...
  m = new MsgData(..);
  ...
  return m;
}
\end{lstlisting}
\end{frame}


\begin{frame}[fragile]
\frametitle{Threaded methods}
 \begin{itemize}
  \item Any method that calls a sync method must be able to suspend:
  \begin{itemize}
   \item Needs to be declared as a \code{threaded} method
   \item A threaded method of a chare C
   \begin{itemize}
    \item Can suspend, without blocking the processor
    \item Other chares can then be executed
    \item Even other methods of chare C can be executed
   \end{itemize}
  \end{itemize}  
 \item Low level thread operations for advanced users:
 \begin{itemize}
   \item \texttt{CthThread CthSelf()}
   \item \texttt{CthAwaken(CthThread t)}
   \item \texttt{CthYield()}
   \item \texttt{CthSuspend()}
 \end{itemize}
\end{itemize}
\end{frame}


\begin{frame}[fragile]
\frametitle{Customized Load Balancers}
\begin{itemize}
\item Statistics collected by Charm
\begin{lstlisting}[basicstyle=\tiny]
struct LDStats { // load balancing database
  ProcStats  *procs; //statistics of PEs
  int count;

  int   n_objs;
  int   n_migrateobjs;
  LDObjData* objData; //info regarding chares

  int   n_comm;
  LDCommData* commData; //communication information

  int  *from_proc, *to_proc; //residence of
  chares
}
\end{lstlisting}
\item Use LDStats, ProcArray and ObjGraph for processor load and communication
statistics
\item \emph{work} is the function invoked by Charm RTS to perform load balancing
\end{itemize}
\end{frame}


\begin{frame}[fragile]
\frametitle{Summary}
\begin{itemize}
 \item Use Profiling and Performance Analysis Tools Early
 \begin{itemize}
  \item \emph{Measure twice, cut once!}
  \item Look for overloaded processors, not underloaded processors
 \end{itemize}
 \item Use PUP for object serialization 
 \begin{itemize}
  \item Enables Migration for Load Balancing or Fault Tolerance
 \end{itemize}
 \item Don't forget to consider granularity
\end{itemize}
\end{frame}
 


\end{document}



